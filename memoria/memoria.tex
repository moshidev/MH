% Adapted from Atanasio Rubio Gil's https://gitlab.com/Groctel/aqademia/-/blob/main/demo/demo_aqademia.tex

\documentclass[10pt, a4paper]{aqademic}

% Language and input encoding

\usepackage[spanish]{babel}

% Document settings

\usepackage[type=CC, modifier=by-nc-sa, version=4.0]{doclicense}
\usepackage{graphicx}
	\graphicspath{{img/}}
\usepackage{multirow}
\usepackage{adjustbox}

\author{Daniel Pedrosa Montes - Grupo A2}
\title{Metaheurísticas. Problema de la Mínima Dispersión Diferencial.}

% Document composition

\begin{document}

\AqMaketitle[%
	cover    = identidad_ugr,
    subtitle = Algoritmos Voraz y Búsqueda Local,
    dni      = {{DNI goes here}},
    email    = {{email goes here}},
	url      = https://github.com/moshidev/MH,
    date     = mayo del 2022
]

\tableofcontents

\chapter{El Problema de la Mínima Dispersión Diferencial}
    
El Problema de la Mínima Dispersión Diferencial es un problema de optimización combinatoria NP-Completo.\cite{Seminario2MH}

Este se presenta en distintas situaciones como al decidir la localización de instalaciones públicas,
la selección de grupos homogéneos, la identificación de grafos densos/regulares o el reparto equitativo en problemas de
flujo de red.\cite{DUARTE201546}

Intuitivamente estamos ante el Problema de la Mínima Dispersión Diferencial cuando dado un conjunto finito de puntos queremos
seleccionar un subconjunto de estos de forma que todos estén más o menos a la misma distancia unos de otros.

\section{Formulación del Problema}

Podemos describir el problema como
$$\textrm{Minimizar } Max_{i\in M}\{\sum_{j\in M}^{}d_{ij}\}-Min_{i\in M}\{\sum_{j\in M}^{}d_{ij}\} $$
Donde $M$ es un subconjunto de $N$, el conjunto de todos los posibles puntos y donde $d_{ij}$ es la distancia desde dos puntos $i$ y $j$.
$|M| = m$, siendo $m$ el número de elementos a escoger de $N$.

\section{Algoritmos para la resolución aproximada en tiempo polinómico}

La complejidad de este problema nos obliga a utilizar algoritmos que puedan encontrar el óptimo de una forma aproximada en un tiempo de computación razonable.
Entre las técnicas más populares podemos destacar GRASP, Búsqueda Local o métodos basados en Poblaciones.\cite{MDP2010}

Nosotros implementaremos entre otros algoritmos de tipo Greedy y de Búsqueda Local. Son algoritmos eficientes de por sí, pero para
mejorar la eficiencia de estos será necesario factorizar el cálculo de las soluciones a partir de otras soluciones de forma
que el algoritmo ejecute el número mínimo posible de operaciones al navegar por el espacio de soluciones.
    
\chapter{Arquitectura de la Solución}
    El objetivo principal del software es el de encontrar soluciones al Problema de la Mínima Dispersión Diferencial lo más cercanas
al óptimo posible y de forma que pueda ejecutarse en un tiempo computacional razonable eficientemente posible. Para conseguir esto
implementamos distintas técnicas de búsqueda, en este caso en concreto utilizando técnicas Greedy y de Búsqueda Local.

\section{Filosofía tras la implementación}

Buscando separar mecanismos de políticas y delimitar claramente las responsabilidades entre clases intentamos generalizar en
distintos módulos lógicos. Después de un proceso iterativo descubrimos que los datos, ya sean de la matriz de distancias
entre puntos o de la solución, son independientes a los algoritmos.
Los algoritmos dependen en estos, pero no al contrario. Por tanto los algoritmos dependen en dos módulos:

    1. Un módulo dedicado a ofrecer mecanismos para importar y acceder a los datos y
    
    2. otro módulo para modificar u obtener información sobre una solución.

Esta separación permite ocultar los detalles acerca de la estructura de datos concreta utilizada tanto como para representar
la información de las distancias entre dos puntos como para representar
la solución al algoritmo. Además, salva a la implementación de este de tener en cuenta todas las operaciones necesarias
para calcular la solución de forma factorizada, pudiendo reutilizarse este código en algoritmos futuros y pudiendo adoptar estructuras
de datos más eficientes en un futuro sin tener por ello que cambiar el código concreto del algoritmo.

Así, los algoritmos describen únicamente las políticas sobre los mecanismos de las estructuras de datos en las que dependen.

\section{Estructuras de datos comunes}

Para representar la matriz que define las distancias entre puntos utilizamos un vector para favorecer la localidad espacial y temporal de la memoria al realizar un acceso secuencial, como cuando se da el caso cuando necesitamos añadir un punto a una solución. Gracias a una función constructura
podemos leer estos datos de un fichero de texto plano, el cual indica tanto las distancias entre puntos como el número de puntos existentes en el
archivo asi como el número de puntos a seleccionar.

Por otra parte para representar una solución utilizamos un diccionario. Hay alternativas más eficientes, como utilizar un vector binario. Por ahora
esto no nos preocupa, ya que las soluciones se obtienen en un tiempo aceptable aun para instancias grandes del problema y al estar abstraidos los mecanismos concretos
para actuar sobre la estructura de datos interna concreta que almacena la solución del algoritmo será fácil de cambiar en un futuro en caso de ser necesario sin
afectar por ello a la implementación de los algoritmos. La interfaz para actuar sobre la solución nos permite añadir y extraer puntos de la solución
de la forma más eficiente posible, además de poder estimar la dispersión de un punto determinado sobre una solución concreta sin tener que añadirlo a
la solución en sí. Para conseguir esto guardamos en el diccionario tanto los puntos escogidos para una solución en concreto como la suma de la distancia desde el punto
al resto de puntos en la solución. En general, estas implementaciones son fruto de unas decisiones de diseño concretas, como la separación de responsabilidades anteriormente
mencionadas, la modificación de un estado únicamente cuando sea necesario, la no pesimización del código y el cálculo eficiente. Además, minimizamos al máximo la copia y la redundancia
de los datos del problema, algoritmo y solución, ya sea permanente o temporal.

También generalizamos la interfaz de los algoritmos de forma que no sea necesario conocer en tiempo de compilación la implementación concreta del algoritmo que resuelva el problema.

\section{Aspectos que comparten los distintos algoritmos}

Ambos algoritmos buscan encontrar los puntos pertenecientes al conjunto dado de tal que formen una solución válida y minimicen la dispersión entre ellos. Por tanto la función objetivo de los algoritmos buscarán encontrar las soluciones tal que $ Disp(Sol^{\prime}) < Disp(Sol) $.

Para tener memoria de los no escogidos en la solución que se devolverá los algoritmos mantienen durante la ejecución del algoritmo un conjunto de puntos no escogidos que se va actualizando cada vez que se añade, retira o intercambia un punto de la solución.

\begin{figure}[h]
    \centering
    \includegraphics[width=\textwidth]{DiagramaClasesArquitectura.drawio.png}
    \caption{Diagrama de Clases de la solución.}
\end{figure}

\chapter{Resolución del Problema de la Mínima Dispersión Diferencial tomando aproximaciones Greedy y de Búsqueda Local}
    Implementamos dos algoritmos que nos ofrecen siempre que la haya una solución válida, más o menos cercana al óptimo.
En concreto un algoritmo de tipo Greedy y otro de Búsqueda Local, caracterizándose respectivamente por ir cogiendo de entre
los puntos no escogidos los que menos dispersión resultan y por partir de una solución inicial aleatoria e ir buscando de
entre los vecinos el primer mejor que mejora la dispersión de la solución que se mantenga en ese momento hasta haber llegado
a un número determinado de iteraciones o hasta no encontrar vecinos que mejoren la dispersión de la solución.

\section{Aproximación Greedy}

Como hemos mencionado anteriormente la estrategia detrás de nuestra aproximación Greedy es la de ir seleccionando
el punto que minimizaría la dispersión de la solución de entre los no seleccionados parando de añadir puntos a la
solución cuando esta fuese una solución válida. Podemos definir el algoritmo en pseudo-código como sigue:

\begin{minipage}{\textwidth}
\begin{lstlisting}[mathescape=true,caption={Definición del algoritmo Greedy.},captionpos=b]
def selecciona_punto_siguiente(solucion, puntos_no_escogidos):
    elemento_siguiente = puntos_no_escogidos[0]
    dispersion_del_elemento_siguiente = $ + \infty  $
    
    por cada punto p en puntos_no_escogidos:
        dispersion = solucion.calcula_dispersión_de_punto_candidato(p)
        si dispersion < dispersion_del_elemento_siguiente:
            elemento_siguiente = p
            devuelve elemento_siguiente
    
    devuelve elemento_siguiente

def solve(numero_de_puntos_a_elegir, mapa_distancias):
    solucion = new Solucion.asocia(mapa_distancias)
    puntos_no_escogidos = new Set.init(Rango(0, mapa_distancias.numero_de_puntos()))

    primer_punto = escoger_aleatorio(puntos_no_escogidos)
    solucion.inserta_punto_en_solucion(primer_punto)
    puntos_no_escogidos.elimina(primer_punto)
    por cada punto a elegir - 1:
        punto_escogido = selecciona_punto_siguiente(solucion, puntos_no_escogidos)
        solucion.inserta_punto_en_solucion(punto_escogido)
        puntos_no_escogidos.elimina(punto_escogido)
    
    devuelve solucion
\end{lstlisting}
\end{minipage}

Recordemos que la inserción y eliminación de un punto se realiza en tiempo $ \Theta (m) $ en una solución
y en tiempo $ O (\log n) $ en un conjunto. El cálculo de la dispersión de un nodo candidato también se realiza
en tiempo $ \Theta (m) $.

\section{Aproximación por Búsqueda Local del Primer Mejor}

El objetivo de los algoritmos de Búsqueda Local es el de examinar exhaustivamente el espacio de soluciones próximo
a una solución determinada. El criterio de parada suele involucrar un máximo de iteraciones o la imposibilidad de encontrar
soluciones entre los vecinos. Es muy bueno por tanto encontrando óptimos, pero generalmente locales debido a la naturaleza del proceso
de generación de soluciones alternativas, ya que únicamente busca entre los vecinos contiguos. Por sí solos este tipo de
algoritmos veremos que no ofrecen soluciones mucho mejores a soluciones de tipo Greedy, y que al igual que estos la
calidad de la solución vendrá muy determinada por el conjunto inicial, que se escoge aleatoriamente.

\begin{minipage}{\textwidth}
\begin{lstlisting}[mathescape=true,caption={Definición del algoritmo de Búsqueda Local.},captionpos=b]
def primer_mejor_vecino_quitando_punto_de_solucion(punto_a_quitar, solucion, puntos_no_escogidos):
    solucion_sin_punto = solucion
    solucion_sin_punto.elimina_punto_de_solucion(punto_a_quitar)

    mejor_punto = punto_a_quitar
    mejor_dispersion = solucion.calcula_dispersion()

    por cada punto p en puntos_no_escogidos:
        dispersion = solucion_sin_punto.calcula_dispersion_de_punto_candidato(p)
        si dispersion < mejor_dispersion:
            mejor_punto = p
            devuelve mejor_punto
    
    devuelve punto_a_quitar

def solve(numero_de_puntos_a_elegir, mapa_distancias):
    solucion = new Solucion.asocia(mapa_distancias)
    puntos_no_escogidos = new Set.init(Rango(0, mapa_distancias.numero_de_puntos()))
    inicializa_solucion_aleatoriamente(solucion, puntos_no_escogidos, numero_de_puntos_a_elegir)

    dispersion = $ + \infty $
    mejor_dispersion = solucion.calcular_dispersion()

    mientras mejor_dispersion < dispersion y num_iteraciones < 100.000:
        dispersion = mejor_dispersion
        por cada punto p en solucion:
            supuesto_mejor_punto = primer_mejor_vecino_quitando_punto_de_solucion(p, solucion, puntos_no_escogidos)
            si supuesto_mejor_punto es distinto de p:
                puntos_no_escogidos.inserta(p)
                puntos_no_escogidos.elimina(supuesto_mejor_punto)
                solucion.elimina_punto_de_solucion(p)
                solucion.inserta_punto_en_solucion(supuesto_mejor_punto)
                mejor_dispersion = solucion.calcula_dispersion()
\end{lstlisting}
\end{minipage}

En el capítulo segundo podemos ver que los métodos que nos proporciona la estructura de datos que hemos declarado y definido para representar y hacer
operaciones hacen que la solución esté factorizada según se indica en los requisitos de la práctica. 

\subsubsection{Métodos de exploración del entorno y operador de generación de vecino}

Como hemos mencionado anteriormente nuestro algoritmo parte de una primera solución aleatoria y a partir de esta empieza
a comparar esta solución con soluciones que va generando pertenecientes al espacio de búsqueda y cercanas a esta. Sustituye
a la solución elegida con el primer mejor vecino que encuentre.

Podemos ver en la función \texttt{primer\_mejor\_vecino\_quitando\_punto\_de\_solucion} del listado 3.2 el algoritmo
para generar soluciones adyacentes dada una solución, de forma que devuelve la primera mejor solución que encuentre.

Intuitivamente se podría describir como si cogiésemos una solución y le quitásemos un punto que pertenezca a esta.
Una vez hecho esto calculamos la dispersión que tendría una posible solución con cada uno de los puntos pertenecientes
al conjunto de los no escogidos. Si encontramos un punto no seleccionado con mejor dispersión devolvemos la solución
con ese punto, en caso contrario devolvemos la solución original. Cabe destacar que esta descripción es aproximada,
no es lo que se describe exactamente en el algoritmo.

\begin{figure}[ht]
    \centering
    \includegraphics[height=7.5cm,keepaspectratio]{busqueda_local_exploracion.png}
    \caption{Generación y selección de vecinos en el algoritmo de Búsqueda Local Primero el Mejor. Las soluciones se generan en el orden 1, 2, 3. En el momento en el que encuentra una solución mejor, la tercera en este caso, la devuelve.}
\end{figure}

\subsubsection{Generación de soluciones aleatorias}

En el listado 3.2 hacemos referencia a una función llamada \texttt{inicializa\_solucion\_aleatoriamente}.
La definimos en el siguiente listado.

\begin{minipage}{\textwidth}
\begin{lstlisting}[mathescape=true,caption={Definición de la función \texttt{inicializa\_solucion\_aleatoriamente} para generar una solución consistente en elementos aleatorios.},captionpos=b]
def inicializa_solucion_aleatoriamente(solucion, puntos_no_escogidos, numero_de_puntos_a_elegir):
    por cada uno de los puntos que tenemos que elegir:
        punto_elegido = escoge_aleatorio(puntos_no_escogidos)
        puntos_no_escogidos.elimina(punto_elegido)
        solucion.inserta_punto_en_solucion(punto_elegido)
\end{lstlisting}
\end{minipage}

\chapter{Resolución del Problema de la Mínima Dispersión Diferencial utilizando Algoritmos Genéticos}
    En la segunda práctica se nos pide implementar un algoritmo genético con distintas variaciones,
según si el esquema de evolución es un esquema generacional con elitismo (AGG) o basada en
uno estacionario (AGE) y también según el operador de cruce, uniforme o basado en posición.
El esquema básico de un algoritmo se caracteriza por, sobre una población previamente inicializada
y en este orden, realizar operaciones sobre la población y con probabilidades preestablecidas
de seleccionar, cruzar mutar y reemplazar en cada iteración del algoritmo.

\section{Operadores comunes a los distintos algoritmos}

Como hemos introducido anteriormente, podemos definir una iteración de un algoritmo genético básico en pseudo-código
con los siguientes pasos:

\begin{minipage}{\textwidth}
\begin{lstlisting}[mathescape=true,caption={Esquema general de un algoritmo genético.},captionpos=b]
var cantidad_cromosomas_a_seleccionar
var num_genes
var probabilidad_cruce
var probabilidad_mutacion

poblacion = generar_poblacion_aleatoria(cantidad_cromosomas_a_seleccionar, num_genes)

def iteracion_algoritmo_genetico(poblacion):
	seleccionados = seleccionar(poblacion, cantidad_cromosomas_a_seleccionar)
	cruzar(seleccionados, probabilidad_cruce)
	mutar(seleccionados, probabilidad_mutacion)
	reemplazar(poblacion, seleccionados)
\end{lstlisting}
\end{minipage}

La definición de los algoritmos para cada uno de estos pasos puede ser diferente según el tipo de
algoritmo genético que diseñemos o implementemos. En este capítulo describiremos nuestra solución
y las diferentes definiciones de funciones para el algoritmo según las variaciones que se nos piden
para este.

\begin{minipage}{\textwidth}

	\subsection{Función de Selección}

	Utilizaremos para las distintas variantes selección por torneo binario. La selección por torneo binario
	consiste en escoger dos cromosomas de la población original al azar, compararlos entre sí y seleccionar el
	que mejor valoración tenga (en nuestro caso la solución con menor dispersión). Esto lo hacemos $n$ veces.
	
	En las variantes generacionales seleccionamos tantos como sea el tamaño de la población original, y en los
	estacionarios únicamente seleccionamos dos.
	
\begin{lstlisting}[mathescape=true,caption={Definición de la función de selección.},captionpos=b]
var generador_aleatorio_uniforme en rango {0, poblacion.size()-1}

def selecciona(poblacion, cantidad_a_seleccionar) -> poblacion_seleccionada:
	poblacion_seleccionada = []
	mientras que seleccionados < cantidad_a_seleccionar
		a = generador_aleatorio_uniforme.siguiente_numero()
		b = generador_aleatorio_uniforme.siguiente_numero()
		poblacion_seleccionada.anexa(a.dispersion < b.dispersion ? a : b)
	devuelve poblacion_seleccionada
\end{lstlisting}
\end{minipage}

\subsection{Función de cruce}

La función de cruce se encarga de seleccionar aleatoriamente parejas de soluciones a cruzar
con una probabilidad determinada. El cruce entre estas dos parejas se podrá realizar de distintas
formas, según veremos en secciones anteriores. El esquema general de la función de cruce es el siguiente.

\begin{minipage}{\textwidth}
\begin{lstlisting}[mathescape=true,caption={Esquema general de la función de cruce.},captionpos=b]
def cruce(poblacion, probabilidad, metodo_cruce):
	por cada pareja de la poblacion escogidas aleatoriamente:
		hijos = metodo_cruce(pareja.a, pareja.b)
		poblacion[pareja.a] = hijos.a
		poblacion[pareja.b] = hijos.b
\end{lstlisting}
\end{minipage}

Generar las parejas de forma aleatoria es un proceso muy costoso. En nuestro caso, ya que
anteriormente la selección se realizó aleatoriamente por torneo binario, no será necesario
volver a aleatorizar el vector que contiene la población de nuevo.

\begin{minipage}{\textwidth}
\begin{lstlisting}[mathescape=true,caption={Definición de la función de cruce.},captionpos=b]
def cruce(poblacion, probabilidad, metodo_cruce):
	cruces_por_hacer = probabilidad * poblacion.size()/2	# Esperanza matematica
	i = 0
	mientras i < cruces_por_hacer:
		# Cruzamos las parejas contiguas en el vector
		hijos = metodo_cruce(poblacion[i*2], poblacion[i*2+1])
		poblacion[i*2] = hijos.a
		poblacion[i*2+1] = hijos.b
		i = i + 1
\end{lstlisting}
\end{minipage}

\subsection{Función de mutación}

Es el tercer paso del esquema general de nuestro algoritmo genético y el que utilizaremos para las
distintas versiones que implementamos. Su función es modificar aleatoriamente algunos genes de
cada uno de los cromosomas para evitar que las soluciones se queden atascadas en zonas concretas
del espacio de búsqueda.

Nuestro operador de mutación selecciona cromosomas al azar, sobre los cuales selecciona uno de
sus genes para sustituirlo por otro al azar de entre los no pertenecientes a la solución (cromosoma).
Como hicimos en la función de cruce nos serviremos de la esperanza matemática para evitar generar números
aleatorios innecesarios.

La idea es la del siguiente listado.

\begin{minipage}{\textwidth}
\begin{lstlisting}[mathescape=true,caption={Definición de la función de mutación.},captionpos=b]
var numero_genes_por_solucion_valida
var dist_aleat_cromosoma en rango {0, poblacion.size()-1}
var dist_aleat_gen en rango {0, numero_genes_por_solucion_valida-1}

def muta(poblacion, probabilidad):
	# Esperanza matematica
	numero_mutaciones = probabilidad * numero_genes_por_solucion_valida
	
	por cada mutacion que tengamos que hacer:
		cromosoma_a_mutar = poblacion[dist_aleat_cromosoma.siguiente()]
		gen_a_mutar = dist_aleat_gen.siguiente()
		cromosoma_a_mutar.elimina(gen_a_mutar)
		cromosoma_a_mutar.inserta(gen aleatorio $\notin$ cromosoma_a_mutar)
\end{lstlisting}
\end{minipage}

\subsection{Función de reemplazamiento}

La función de reemplazamiento cambia completamente según la variación del algoritmo que hayamos
implementado. En secciones siguientes explicaremos la definición concreta del algoritmo que hayamos
utilizado.

El objetivo de la función de reemplazamiento es el de sustituir la población del inicio de la iteración
del algoritmo genético (o partes de esta) por la población modificada por los operadores vistos en subsecciones
anteriores (o partes de esta).

\section{Operadores de cruce}

Como dijimos anteriormente, la función de cruce se encarga de seleccionar aleatoriamente parejas de soluciones a cruzar
con una probabilidad determinada. Existen distintas técnicas para cruzar dos cromosomas.
Nosotros implementamos dos métodos distintos para cruzar dos soluciones, uno uniforme y otro basado en posición.

\subsection{Operador de cruce uniforme}

En nuestro problema, el MDD, el operador de cruce uniforme busca conservar las selecciones prometedoras manteniendo
la intersección de genes de ambos padres (tanto de genes en común como genes que no tienen ninguno de los dos). Para el
resto de selecciones que escapan la intersección los hijos se crean eligiendo aleatoriamente genes de un padre o del otro.

\begin{minipage}{\textwidth}
\begin{lstlisting}[mathescape=true,caption={Definición del operador de cruce uniforme.},captionpos=b]
def cruce_uniforme(padre_a, padre_b) -> hijo_a,hijo_b:
	hijo_a = padre_a $\cap $ padre_b
	hijo_b = padre_a $\cap $ padre_b
	por cada hijo:
		por cada gen no perteneciente a padre_a $\cap $ padre_b:	# padre_a $xor$ padre_b
			hijo.inserta_gen_al_50%_de_probabilidad(gen)
		repara(hijo)

	devuelve hijo_a,hijo_b
\end{lstlisting}
\end{minipage}

\begin{minipage}{\textwidth}

\subsubsection{Reparador para el operador uniforme}

Este operador genera soluciones que pueden no ser factibles, por tanto necesitaremos una función que repare a los hijos que genere
nuestro operador de cruce.

\begin{lstlisting}[mathescape=true,caption={Definición del operador de reparación para el operador de cruce uniforme.},captionpos=b]
def repara(cromosoma):
	si el cromosoma es factible:
		terminar
	mientras que el cromosoma tenga $mas$ genes de los que debería:
		gen = gen_perteneciente_al_cromosoma_que_maximiza_la_dispersion(cromosoma)
		cromosoma.elimina(gen)
	mientras que el cromosoma tenga $menos$ genes de los que debería:
		gen = gen_no_escogido_en_cromosoma_que_minimiza_la_dispersion(cromosoma)
		cromosoma.inserta(gen)
\end{lstlisting}
\end{minipage}

\subsection{Operador de cruce basado en posición}

El operador de cruce basado en posición al igual que el uniforme conserva en los hijos la intersección de los padres.
Se diferencia del uniforme en que en vez de ir escogiendo genes para los hijos aleatoriamente entre los padres distribuye
los genes de los padres que no intersectan entre los dos hijos (y por tanto no necesitando una función de reparación).
Es más disruptivo que el uniforme, y por tanto es más complicado que converja. \cite{Seminario3MH}
Definimos el operador de la siguiente forma.

\begin{minipage}{\textwidth}
\begin{lstlisting}[mathescape=true,caption={Definición del operador de cruce basado en posición.},captionpos=b]
def cruce_posicion(padre_a, padre_b) -> hijo_a,hijo_b:
	hijo_a = padre_a $\cap $ padre_b
	hijo_b = padre_a $\cap $ padre_b
	gen_xor = padre_a $\cup $ padre_b - padre_a $\cap $ padre_b
	aleatoriza(gen_xor)
	
	hijo_a.inserta(primera mitad gen_xor)
	hijo_b.inserta(segunda mitad gen_xor)

	devuelve hijo_a,hijo_b
\end{lstlisting}
\end{minipage}

\section{Modelos de los algoritmos genéticos implementados}

\subsection{Modelo generacional}

Consiste en seleccionar por torneo binario tantos cromosomas como cromosomas tenga la población original, cruzar
con una probabilidad $P_{c}$ (que en nuestro caso será de 0.7), mutar esta selección cruzada
una probabilidad $P_{m}$ (en nuestro caso 0.1) y reemplazar la población original por la nueva, aunque con elitismo,
esto es, manteniendo la mejor solución de la población original sobre el peor cromosoma de la nueva.
Definimos el operador de reemplazamiento del modelo generacional en el siguiente listado.

\begin{minipage}{\textwidth}
\begin{lstlisting}[mathescape=true,caption={Definición del operador de reemplazamiento del modelo generacional.},captionpos=b]
def reemplazamiento_generacional(a_reemplazar, poblacion_nueva):
	mejor_solucion = encuentra_mejor_solucion(a_reemplazar)
	swap(mejor_solucion, poblacion_nueva)
	a_reemplazar.elimina(encuentra_peor_solucion(poblacion_nueva))
	a_reemplazar.inserta(mejor_solucion)
\end{lstlisting}
\end{minipage}

\subsection{Modelo estacionario}

Por otra parte en el modelo estacionario en cada iteración del algoritmo genético seleccionamos únicamente dos
cromosomas, también por torneo binario. Cruzamos los cromosomas seleccionados con probabilidad 1 y mutamos con
probabilidad $P_{m}$ (en nuestro caso 0.1). A la hora de reemplazar la población original por la nueva únicamente
reemplazamos las peores soluciones de la original por las de la nueva únicamente si son mejores que esta. Podemos
verlo en la siguiente definición del algoritmo de reemplazamiento.

\begin{minipage}{\textwidth}
\begin{lstlisting}[mathescape=true,caption={Definición del operador de reemplazamiento del modelo estacionario.},captionpos=b]
def reemplazamiento_estacionario(a_reemplazar, poblacion_nueva):
	por cada cromosoma de la poblacion_nueva:
		peor_solucion = encuentra_peor_solucion(poblacion_nueva)
		si cromosoma.dispersion < peor_solucion.dispersion
			a_reemplazar.elimina(peor_solucion)
			a_reemplazar.inserta(cromosoma)
\end{lstlisting}
\end{minipage}

\chapter{Resolución del Problema de la Mínima Dispersión Diferencial utilizando Algoritmos Meméticos}
    Los algoritmos meméticos hibridan los algoritmos de búsqueda local con genéticos para obtener
lo mejor de ambos, la exhaustividad de búsqueda en una parte concreta del espacio de soluciones
y la amplitud de búsqueda en el espacio de soluciones respectivamente. Podemos jugar con la
proporción de tiempo y la cantidad de cromosomas que evolucionamos o sometemos a búsqueda local
para intentar obtener resultados diferentes.

\section{Esquema de un Algoritmo Memético}

Para la BL utilizamos un algoritmo de búsqueda local por mejor vecino con un máximo de 400 vecinos a evaluar
y para el algoritmo genético utilizamos un AGG-uniforme.

\begin{minipage}{\textwidth}
\begin{lstlisting}[mathescape=true,caption={Esquema general de un algoritmo memético.},captionpos=b]
var cantidad_cromosomas_a_seleccionar
var num_genes
var probabilidad_cruce
var probabilidad_mutacion
var algoritmo_genetico(probabilidad_cruce, probabilidad_mutacion)

var algoritmo_bl
var limite_vecinos_bl

var porcentaje_cromosomas_a_someter_a_busqueda_local
var numero_generaciones_entre_busquedas_locales
var ordenar_seleccion_para_bl

poblacion =
algoritmo_genetico.inicializar_poblacion_aleatoriamente(cantidad_cromosomas, num_genes)

def iteracion_algoritmo_memetico(poblacion):
	repite numero_generaciones_entre_busquedas_locales:
		algoritmo_genetico.iteracion(poblacion)

	seleccionados = selecciona_para_bl(poblacion, porcentaje_cromosomas_a_someter_a_busqueda_local, ordenar_seleccion_para_bl)
	por cada cromosoma c en seleccionados:
		poblacion.elimina(c)
		res = algoritmo_bl.resuelve(c, num_genes, limite_vecinos_bl)
		poblacion.inserta(res)
\end{lstlisting}
\end{minipage}

Dónde la función para seleccionar los cromosomas que se someterán a búsqueda local se define como:

\begin{minipage}{\textwidth}
\begin{lstlisting}[mathescape=true,caption={Definición de la función selecciona\_para\_bl.},captionpos=b]
def selecciona_para_bl(poblacion, porcentaje_a_seleccionar, ordenar_seleccion):
	seleccionados = []

	si ordenar_seleccion:
		ordena(poblacion)
	si no:
		aleatoriza(poblacion)
	
	seleccionados.inserta(poblacion.begin(), poblacion.end()+porcentaje_a_seleccionar*poblacion.size())
	población.elimina(poblacion.begin(), poblacion.end()+porcentaje_a_seleccionar*poblacion.size())

	devuelve seleccionados
\end{lstlisting}
\end{minipage}

\chapter{Obtención y análisis de resultados}
    Para analizar la calidad del algoritmo realizamos dos tipos diferentes de benchmark, uno que cuantifica la dispersión
media de distintas ejecuciones con distintas semillas para cada algoritmo y otro que estima después de un número
determinado de ejecuciones el tiempo de ejecución medio de cada algoritmo para cada dataset. Cuanto más se aproxime
a cero la dispersión mejor será el algoritmo. A iguales dispersiones consideraremos que un algoritmo es mejor que otro
si el tiempo de ejecución es significativamente menor.

Para el caso de los algoritmos genéticos, a diferencia de para los casos de los algoritmos Greedy y BL, recogeremos 
por cada fichero de benchmark la solución que nos proporcione el algoritmo en cuestión, para una única semilla. Procedemos
de igual forma para el caso de los algoritmos meméticos.

\section{Procedimiento para la obtención de resultados}

Los resultados disponibles en los cuadros de esta sección han sido resultado de distintas ejecuciones.

Para obtener la dispersión hemos ejecutado el algoritmo para cada dataset para 5 semillas distintas disponibles
en el script lanzador. Cada fila de los cuadros 3 y 4 corresponde a la dispersión media de los 5 resultados de cada
una de estas ejecuciones (para los algoritmos que corresponda)

Por otra parte, para obtener el tiempo medio de ejecución para fila de las tablas hemos ejecutado cada dataset
para 5 semillas unas 50 veces en invocaciones diferentes y obtenido la media de los resultados del tiempo de estas
$ 50 \cdot 5 $ ejecuciones (para los algoritmos que corresponda)

Por último hemos calculado la desviación media de las soluciones para cada tabla como la suma de las desviaciones
entre el número de casos, símil para el caso del cálculo del tiempo medio. La desviación la hemos calculado como la
diferencia entre el coste en dispersión de nuestro algoritmo y el coste de nuestra referencia.

\begin{table}[!ht]%
    \centering
    \begin{tabular}{|l|l|l|}
        \hline
        \textbf{Algoritmo} & \textbf{Desviación~Media} & \textbf{Tiempo~(s)} \\ \hline
        Greedy & 129,45 & 1,53E-03 \\ \hline
        Búsqueda~Local~Primer~Mejor & 87,28 & 1,57E-02 \\ \hline
        Búsqueda~Local & 97,79 & 1,56E-02 \\ \hline
    \end{tabular}
\caption{Tabla resumen de las ejecuciones.}
\end{table}

\section{Análisis de los datos obtenidos}

En general, los resultados obtenidos son bastante mejorables si los comparamos con las dispersiones de referencia. Como hemos
comentado anteriormente y volveremos a repetir, el factor que más influye en la obtención de resultados es la solución inicial,
ya que al al explorar el espacio de soluciones casi que exclusivamente inmediatamente contiguo será fácil llegar a una solución
la cual no tenga vecinos mejores, quedando por tanto atrapado en un óptimo local y ofreciendo costes muy diferentes según la semilla
utilizada para el generador aleatorio.

Podemos observar estos hechos en la figura 6.1, especialemente en el caso de la Búsqueda Local por Primer Mejor Vecino y por Mejor Vecino.
En general podemos observar que existen muchos resultados con dispersiones muy distintas por encima del tercer cuartil. También lo explicamos
como consecuencia de los factores expuestos.

Destaca el algoritmo de Búsqueda Local por Mejor Vecino por dar peores resultados realizando más iteraciones que el de por Primer
Mejor Vecino, aunque con mejores resultados de media que el Greedy y con la mediana prácticamente idéntica al Búsqueda Local por Primer Mejor Vecino.
Esto se debe a que aunque recorre más exhaustivamente el espacio de soluciones y la limitación operador de mutación seguido la BL por Mejor Vecino
se quede atascado en un óptimo local sin haber explorado mucho el espacio de soluciones, al igual que le ocurre de forma parecida al algoritmo Greedy.

\begin{figure}[ht]
    \centering
    \includegraphics[keepaspectratio,width=\textwidth]{box_and_whisker_dispersion_media_dataset_y_tipo_algoritmo.png}
    \caption{Dispersión media por dataset y por tipo de algoritmo. Podemos observar que hay una alta desviación típica del coste de las soluciones.
    Podemos ver que una peor media de resultados en las ejecuciones no implica una peor mediana, y una peor mediana no implica un intervalo del percentil 25 al 75 peor.
    Destacamos también la correlación entre medianas dentro de un mismo tipo de algoritmo.}
\end{figure}

Para aumentar el espacio en el espacio de soluciones del conjunto de soluciones consideradas por los algoritmos Greedy y de Búsqueda Local y no quedar
así atrapados en óptimos locales utilizando estrategias de BL podríamos probar variantes como la Búsqueda Local Estocástica, Búsqueda Local por Primer Mejor Aleatorio, Búsqueda Local
con Reinicio Aleatorio u otras técnicas más avanzadas como el Enfriamiento Simulado o búsqueda \textit{Local Beam} entre otros.\cite[Sección 4.1]{russell2020artificial}

Otra opción es la de usar algoritmos genéticos. Podemos ver que ofrecen unos muy buenos resultados, gracias a que no sólo van saltando más por el espacio de soluciones
sino que además van conservando las partes de las soluciones más prometedoras, alejándolos de una simple búsqueda aleatoria. Podemos ver cómo para el problema del MDD los modelos
estacionarios dan mejores soluciones que los generacionales. Esto se puede deber a que los generacionales sustituyen completamente la población original, mientras que los estacionarios
únicamente inserta las nuevas soluciones en la población inicial si estas son mejores, conservando más información acerca de las soluciones más prometedoras y no cayendo tanto en la aleatoriedad.

De igual forma podemos destacar cómo el operador de cruce uniforme ofrece mejores resultados que el de posición. Esto se debe a que ambos cruces, aunque mantienen la intersección de genes,
aleatorizan los genes que quedan por escoger. Sin embargo el cruce uniforme, al reparar la solución, va buscando y eliminando los genes que más le convenga, aplicando un esquema Greedy, que
seguramente ayudará a obtener soluciones con mejor fitness.

En nuestras ejecuciones los algoritmos meméticos, aunque mejores que una búsqueda local, dan peor rendimiento que el resto de algoritmos genéticos.
En este, cuanto más peso le damos al paso genético y cuanto más fuerte sea el elitismo, da mejores resultados.

\pagebreak

\begin{table}[!ht]%
    \centering    
    \begin{adjustbox}{height=12cm}
    \begin{tabular}{|l|l|l|l|}
    \hline
        Caso & Coste~Medio~Obtenido & Desv & Tiempo~(s) \\ \hline
        GKD-b\_1\_n25\_m2 & 0,0000 & 0,00 & 1,22E-05 \\ \hline
        GKD-b\_2\_n25\_m2 & 0,0000 & 0,00 & 1,36E-05 \\ \hline
        GKD-b\_3\_n25\_m2 & 0,0000 & 0,00 & 1,31E-05 \\ \hline
        GKD-b\_4\_n25\_m2 & 0,0000 & 0,00 & 1,17E-05 \\ \hline
        GKD-b\_5\_n25\_m2 & 0,0000 & 0,00 & 1,14E-05 \\ \hline
        GKD-b\_6\_n25\_m7 & 70,9064 & 58,19 & 5,11E-05 \\ \hline
        GKD-b\_7\_n25\_m7 & 57,0336 & 42,93 & 4,77E-05 \\ \hline
        GKD-b\_8\_n25\_m7 & 44,6512 & 27,89 & 5,08E-05 \\ \hline
        GKD-b\_9\_n25\_m7 & 57,4252 & 40,36 & 5,40E-05 \\ \hline
        GKD-b\_10\_n25\_m7 & 73,3526 & 50,09 & 5,98E-05 \\ \hline
        GKD-b\_11\_n50\_m5 & 20,9113 & 18,99 & 6,85E-05 \\ \hline
        GKD-b\_12\_n50\_m5 & 25,0564 & 22,94 & 7,73E-05 \\ \hline
        GKD-b\_13\_n50\_m5 & 24,8575 & 22,50 & 8,35E-05 \\ \hline
        GKD-b\_14\_n50\_m5 & 30,4310 & 28,77 & 8,29E-05 \\ \hline
        GKD-b\_15\_n50\_m5 & 39,1907 & 36,34 & 8,00E-05 \\ \hline
        GKD-b\_16\_n50\_m15 & 158,8222 & 116,08 & 4,10E-04 \\ \hline
        GKD-b\_17\_n50\_m15 & 150,3960 & 102,29 & 4,11E-04 \\ \hline
        GKD-b\_18\_n50\_m15 & 156,8830 & 113,69 & 4,32E-04 \\ \hline
        GKD-b\_19\_n50\_m15 & 151,8338 & 105,42 & 3,96E-04 \\ \hline
        GKD-b\_20\_n50\_m15 & 145,6224 & 97,91 & 3,48E-04 \\ \hline
        GKD-b\_21\_n100\_m10 & 78,0183 & 64,19 & 3,79E-04 \\ \hline
        GKD-b\_22\_n100\_m10 & 76,9076 & 63,24 & 3,50E-04 \\ \hline
        GKD-b\_23\_n100\_m10 & 81,3711 & 66,03 & 3,44E-04 \\ \hline
        GKD-b\_24\_n100\_m10 & 79,6570 & 71,02 & 3,68E-04 \\ \hline
        GKD-b\_25\_n100\_m10 & 56,6460 & 39,45 & 3,43E-04 \\ \hline
        GKD-b\_26\_n100\_m30 & 377,4836 & 208,75 & 1,98E-03 \\ \hline
        GKD-b\_27\_n100\_m30 & 446,3368 & 319,24 & 2,08E-03 \\ \hline
        GKD-b\_28\_n100\_m30 & 384,7054 & 278,33 & 2,01E-03 \\ \hline
        GKD-b\_29\_n100\_m30 & 273,4050 & 135,95 & 1,98E-03 \\ \hline
        GKD-b\_30\_n100\_m30 & 393,1588 & 265,68 & 2,08E-03 \\ \hline
        GKD-b\_31\_n125\_m12 & 70,1099 & 58,36 & 5,46E-04 \\ \hline
        GKD-b\_32\_n125\_m12 & 89,5372 & 70,75 & 5,56E-04 \\ \hline
        GKD-b\_33\_n125\_m12 & 88,0445 & 69,51 & 5,52E-04 \\ \hline
        GKD-b\_34\_n125\_m12 & 73,3446 & 53,86 & 6,36E-04 \\ \hline
        GKD-b\_35\_n125\_m12 & 95,3074 & 77,19 & 6,45E-04 \\ \hline
        GKD-b\_36\_n125\_m37 & 468,0492 & 312,61 & 3,97E-03 \\ \hline
        GKD-b\_37\_n125\_m37 & 500,5334 & 301,64 & 4,27E-03 \\ \hline
        GKD-b\_38\_n125\_m37 & 553,3864 & 365,42 & 4,17E-03 \\ \hline
        GKD-b\_39\_n125\_m37 & 479,3044 & 310,71 & 4,37E-03 \\ \hline
        GKD-b\_40\_n125\_m37 & 501,5562 & 323,36 & 4,25E-03 \\ \hline
        GKD-b\_41\_n150\_m15 & 113,3785 & 90,03 & 1,14E-03 \\ \hline
        GKD-b\_42\_n150\_m15 & 144,4201 & 117,63 & 1,09E-03 \\ \hline
        GKD-b\_43\_n150\_m15 & 116,8756 & 90,12 & 1,05E-03 \\ \hline
        GKD-b\_44\_n150\_m15 & 109,1999 & 83,26 & 1,04E-03 \\ \hline
        GKD-b\_45\_n150\_m15 & 102,7905 & 75,02 & 1,01E-03 \\ \hline
        GKD-b\_46\_n150\_m45 & 637,9208 & 410,17 & 6,46E-03 \\ \hline
        GKD-b\_47\_n150\_m45 & 477,0816 & 248,48 & 6,39E-03 \\ \hline
        GKD-b\_48\_n150\_m45 & 609,1548 & 382,41 & 6,57E-03 \\ \hline
        GKD-b\_49\_n150\_m45 & 639,2372 & 412,83 & 6,60E-03 \\ \hline
        GKD-b\_50\_n150\_m45 & 471,5032 & 222,65 & 6,38E-03 \\ \hline
    \end{tabular}
    \end{adjustbox}
    \caption{Resultados de la ejecución del algoritmo Greedy}
\end{table}

\pagebreak
\input{resources/tabla_ejecuciones_localsearchbestfirst.tex}
\pagebreak

\begin{table}[!ht]
    \centering
    \begin{adjustbox}{height=12cm}
    \begin{tabular}{|l|l|l|l|}
    \hline
        Caso & Coste medio obtenido & Desv & Tiempo \\ \hline
        GKD-b\_1\_n25\_m2 & 0,0000 & 0,00 & 1,86E-05 \\ \hline
        GKD-b\_2\_n25\_m2 & 0,0000 & 0,00 & 1,70E-05 \\ \hline
        GKD-b\_3\_n25\_m2 & 0,0000 & 0,00 & 1,89E-05 \\ \hline
        GKD-b\_4\_n25\_m2 & 0,0000 & 0,00 & 1,76E-05 \\ \hline
        GKD-b\_5\_n25\_m2 & 0,0000 & 0,00 & 1,72E-05 \\ \hline
        GKD-b\_6\_n25\_m7 & 30,5731 & 17,86 & 2,73E-04 \\ \hline
        GKD-b\_7\_n25\_m7 & 27,2974 & 13,20 & 2,16E-04 \\ \hline
        GKD-b\_8\_n25\_m7 & 42,2687 & 25,51 & 2,41E-04 \\ \hline
        GKD-b\_9\_n25\_m7 & 40,2675 & 23,20 & 2,42E-04 \\ \hline
        GKD-b\_10\_n25\_m7 & 45,7940 & 22,53 & 2,23E-04 \\ \hline
        GKD-b\_11\_n50\_m5 & 15,5196 & 13,59 & 2,80E-04 \\ \hline
        GKD-b\_12\_n50\_m5 & 11,3000 & 9,18 & 2,35E-04 \\ \hline
        GKD-b\_13\_n50\_m5 & 13,1555 & 10,79 & 2,84E-04 \\ \hline
        GKD-b\_14\_n50\_m5 & 21,0886 & 19,43 & 2,26E-04 \\ \hline
        GKD-b\_15\_n50\_m5 & 17,2445 & 14,39 & 2,65E-04 \\ \hline
        GKD-b\_16\_n50\_m15 & 100,7846 & 58,04 & 2,99E-03 \\ \hline
        GKD-b\_17\_n50\_m15 & 115,8601 & 67,75 & 3,04E-03 \\ \hline
        GKD-b\_18\_n50\_m15 & 87,7141 & 44,52 & 2,18E-03 \\ \hline
        GKD-b\_19\_n50\_m15 & 110,8314 & 64,42 & 2,26E-03 \\ \hline
        GKD-b\_20\_n50\_m15 & 112,7244 & 65,01 & 2,84E-03 \\ \hline
        GKD-b\_21\_n100\_m10 & 33,8073 & 19,98 & 2,09E-03 \\ \hline
        GKD-b\_22\_n100\_m10 & 47,3694 & 33,71 & 1,76E-03 \\ \hline
        GKD-b\_23\_n100\_m10 & 38,7309 & 23,39 & 1,94E-03 \\ \hline
        GKD-b\_24\_n100\_m10 & 37,8747 & 29,23 & 2,02E-03 \\ \hline
        GKD-b\_25\_n100\_m10 & 36,5649 & 19,36 & 1,96E-03 \\ \hline
        GKD-b\_26\_n100\_m30 & 383,0796 & 214,35 & 1,40E-02 \\ \hline
        GKD-b\_27\_n100\_m30 & 319,3224 & 192,23 & 2,10E-02 \\ \hline
        GKD-b\_28\_n100\_m30 & 259,7818 & 153,40 & 2,38E-02 \\ \hline
        GKD-b\_29\_n100\_m30 & 290,5872 & 153,13 & 2,09E-02 \\ \hline
        GKD-b\_30\_n100\_m30 & 303,7936 & 176,31 & 1,95E-02 \\ \hline
        GKD-b\_31\_n125\_m12 & 48,1569 & 36,41 & 3,34E-03 \\ \hline
        GKD-b\_32\_n125\_m12 & 47,2372 & 28,45 & 3,30E-03 \\ \hline
        GKD-b\_33\_n125\_m12 & 43,2685 & 24,74 & 3,68E-03 \\ \hline
        GKD-b\_34\_n125\_m12 & 48,4522 & 28,96 & 3,77E-03 \\ \hline
        GKD-b\_35\_n125\_m12 & 57,8748 & 39,76 & 3,19E-03 \\ \hline
        GKD-b\_36\_n125\_m37 & 341,0166 & 185,58 & 4,36E-02 \\ \hline
        GKD-b\_37\_n125\_m37 & 524,9386 & 326,04 & 4,93E-02 \\ \hline
        GKD-b\_38\_n125\_m37 & 514,5118 & 326,54 & 4,93E-02 \\ \hline
        GKD-b\_39\_n125\_m37 & 352,4736 & 183,88 & 2,90E-02 \\ \hline
        GKD-b\_40\_n125\_m37 & 388,1836 & 209,99 & 3,89E-02 \\ \hline
        GKD-b\_41\_n150\_m15 & 66,5911 & 43,25 & 7,64E-03 \\ \hline
        GKD-b\_42\_n150\_m15 & 72,0939 & 45,30 & 9,57E-03 \\ \hline
        GKD-b\_43\_n150\_m15 & 75,4758 & 48,72 & 1,07E-02 \\ \hline
        GKD-b\_44\_n150\_m15 & 68,2100 & 42,27 & 7,63E-03 \\ \hline
        GKD-b\_45\_n150\_m15 & 68,9531 & 41,18 & 6,29E-03 \\ \hline
        GKD-b\_46\_n150\_m45 & 529,0926 & 301,34 & 9,12E-02 \\ \hline
        GKD-b\_47\_n150\_m45 & 393,5756 & 164,97 & 7,03E-02 \\ \hline
        GKD-b\_48\_n150\_m45 & 485,8222 & 259,08 & 8,67E-02 \\ \hline
        GKD-b\_49\_n150\_m45 & 504,2588 & 277,85 & 6,70E-02 \\ \hline
        GKD-b\_50\_n150\_m45 & 513,8540 & 265,00 & 7,62E-02 \\ \hline
    \end{tabular}
    \end{adjustbox}
    \caption{Resultados de la ejecución del algoritmo de Búsqueda Local Primero el Mejor}
\end{table}

\pagebreak
\pagebreak

\begin{table}[!ht]%
    \centering    
    \begin{adjustbox}{height=12cm}
    \begin{tabular}{|l|l|l|l|}
    \hline
        Caso & Coste~Medio~Obtenido & Desv & Tiempo~(s) \\ \hline
		GKD-b\_1\_n25\_m2    & 0       & 0         & 0,0608214 \\ \hline
		GKD-b\_2\_n25\_m2    & 0       & 0         & 0,0601605 \\ \hline
		GKD-b\_3\_n25\_m2    & 0       & 0         & 0,0601253 \\ \hline
		GKD-b\_4\_n25\_m2    & 0       & 0         & 0,0610055 \\ \hline
		GKD-b\_5\_n25\_m2    & 0       & 0         & 0,0614487 \\ \hline
		GKD-b\_6\_n25\_m7    & 13,4793 & 0,76134   & 0,290592  \\ \hline
		GKD-b\_7\_n25\_m7    & 14,0988 & 0     & 0,286976  \\ \hline
		GKD-b\_8\_n25\_m7    & 16,7612 & 0     & 0,27321   \\ \hline
		GKD-b\_9\_n25\_m7    & 21,7391 & 4,66989   & 0,270595  \\ \hline
		GKD-b\_10\_n25\_m7   & 26,238  & 2,97277   & 0,305522  \\ \hline
		GKD-b\_11\_n50\_m5   & 1,92609 & 0    & 0,202758  \\ \hline
		GKD-b\_12\_n50\_m5   & 2,91382 & 0,79278   & 0,207368  \\ \hline
		GKD-b\_13\_n50\_m5   & 3,46385 & 1,10154   & 0,202296  \\ \hline
		GKD-b\_14\_n50\_m5   & 5,22367 & 3,56047   & 0,199245  \\ \hline
		GKD-b\_15\_n50\_m5   & 4,9415  & 2,08837   & 0,205534  \\ \hline
		GKD-b\_16\_n50\_m15  & 72,3744 & 29,62862  & 1,35839   \\ \hline
		GKD-b\_17\_n50\_m15  & 59,4868 & 11,37919  & 1,3981    \\ \hline
		GKD-b\_18\_n50\_m15  & 53,5759 & 10,37981  & 1,51368   \\ \hline
		GKD-b\_19\_n50\_m15  & 59,2819 & 12,86945  & 1,43361   \\ \hline
		GKD-b\_20\_n50\_m15  & 47,7151 & 0    & 1,30483   \\ \hline
		GKD-b\_21\_n100\_m10 & 30,0202 & 16,18818  & 1,22368   \\ \hline
		GKD-b\_22\_n100\_m10 & 23,1114 & 9,44706   & 0,826281  \\ \hline
		GKD-b\_23\_n100\_m10 & 20,6145 & 5,26912   & 0,750946  \\ \hline
		GKD-b\_24\_n100\_m10 & 22,188  & 13,54736  & 0,855833  \\ \hline
		GKD-b\_25\_n100\_m10 & 21,111  & 3,91049   & 0,828859  \\ \hline
		GKD-b\_26\_n100\_m30 & 298,858 & 130,12841 & 7,84287   \\ \hline
		GKD-b\_27\_n100\_m30 & 270,049 & 142,95174 & 8,089     \\ \hline
		GKD-b\_28\_n100\_m30 & 266,375 & 159,99581 & 7,922     \\ \hline
		GKD-b\_29\_n100\_m30 & 224,208 & 86,75484  & 7,98288   \\ \hline
		GKD-b\_30\_n100\_m30 & 245,881 & 118,40126 & 7,92112   \\ \hline
		GKD-b\_31\_n125\_m12 & 38,3591 & 26,61396  & 1,19129   \\ \hline
		GKD-b\_32\_n125\_m12 & 27,0862 & 8,29727   & 1,22506   \\ \hline
		GKD-b\_33\_n125\_m12 & 22,4956 & 3,964     & 1,23151   \\ \hline
		GKD-b\_34\_n125\_m12 & 35,6394 & 16,15107  & 1,20127   \\ \hline
		GKD-b\_35\_n125\_m12 & 31,1067 & 12,99428  & 1,21964   \\ \hline
		GKD-b\_36\_n125\_m37 & 354,779 & 199,34423 & 14,5411   \\ \hline
		GKD-b\_37\_n125\_m37 & 484,375 & 285,48038 & 14,8467   \\ \hline
		GKD-b\_38\_n125\_m37 & 416,108 & 228,14097 & 14,4475   \\ \hline
		GKD-b\_39\_n125\_m37 & 359,15  & 190,5598  & 14,3384   \\ \hline
		GKD-b\_40\_n125\_m37 & 465,486 & 287,29226 & 14,3901   \\ \hline
		GKD-b\_41\_n150\_m15 & 49,5475 & 26,20142  & 2,19951   \\ \hline
		GKD-b\_42\_n150\_m15 & 37,2519 & 10,4624   & 2,22639   \\ \hline
		GKD-b\_43\_n150\_m15 & 57,8698 & 31,11533  & 2,19587   \\ \hline
		GKD-b\_44\_n150\_m15 & 57,5674 & 31,63181  & 2,16912   \\ \hline
		GKD-b\_45\_n150\_m15 & 44,7964 & 17,02339  & 2,3015    \\ \hline
		GKD-b\_46\_n150\_m45 & 642,16  & 414,41069 & 24,888    \\ \hline
		GKD-b\_47\_n150\_m45 & 511,976 & 283,3731  & 24,4736   \\ \hline
		GKD-b\_48\_n150\_m45 & 560,053 & 333,30766 & 24,6001   \\ \hline
		GKD-b\_49\_n150\_m45 & 589,826 & 363,41639 & 25,2275   \\ \hline
		GKD-b\_50\_n150\_m45 & 581,139 & 332,28238 & 25,9646   \\ \hline
    \end{tabular}
    \end{adjustbox}
    \caption{Resultados de la ejecución del algoritmo \textbf{AGG-uniforme}}
\end{table}

\pagebreak
\pagebreak

\begin{table}[!ht]%
    \centering    
    \begin{adjustbox}{height=12cm}
    \begin{tabular}{|l|l|l|l|}
    \hline
        Caso & Coste~Medio~Obtenido & Desv & Tiempo~(s) \\ \hline
		GKD-b\_1\_n25\_m2    & 0       & 0         & 0,0662028 \\ \hline
		GKD-b\_2\_n25\_m2    & 0       & 0         & 0,0610938 \\ \hline
		GKD-b\_3\_n25\_m2    & 0       & 0         & 0,0580372 \\ \hline
		GKD-b\_4\_n25\_m2    & 0       & 0         & 0,064511  \\ \hline
		GKD-b\_5\_n25\_m2    & 0       & 0         & 0,0620919 \\ \hline
		GKD-b\_6\_n25\_m7    & 13,4793 & 0,76134   & 0,224877  \\ \hline
		GKD-b\_7\_n25\_m7    & 16,743  & 2,64425   & 0,234692  \\ \hline
		GKD-b\_8\_n25\_m7    & 21,8265 & 5,06531   & 0,255093  \\ \hline
		GKD-b\_9\_n25\_m7    & 17,0692 & 0    & 0,264479  \\ \hline
		GKD-b\_10\_n25\_m7   & 26,238  & 2,97277   & 0,249039  \\ \hline
		GKD-b\_11\_n50\_m5   & 1,92609 & 0    & 0,162693  \\ \hline
		GKD-b\_12\_n50\_m5   & 6,87868 & 4,75764   & 0,173385  \\ \hline
		GKD-b\_13\_n50\_m5   & 9,72116 & 7,35885   & 0,159223  \\ \hline
		GKD-b\_14\_n50\_m5   & 5,22367 & 3,56047   & 0,228844  \\ \hline
		GKD-b\_15\_n50\_m5   & 4,9415  & 2,08837   & 0,183588  \\ \hline
		GKD-b\_16\_n50\_m15  & 42,7458 & 0     & 0,761205  \\ \hline
		GKD-b\_17\_n50\_m15  & 84,5719 & 36,46429  & 0,730784  \\ \hline
		GKD-b\_18\_n50\_m15  & 66,1468 & 22,95071  & 0,677748  \\ \hline
		GKD-b\_19\_n50\_m15  & 53,8508 & 7,43835   & 0,717973  \\ \hline
		GKD-b\_20\_n50\_m15  & 88,5122 & 40,79709  & 0,730743  \\ \hline
		GKD-b\_21\_n100\_m10 & 29,2964 & 15,46438  & 0,424359  \\ \hline
		GKD-b\_22\_n100\_m10 & 32,0481 & 18,38376  & 0,39522   \\ \hline
		GKD-b\_23\_n100\_m10 & 37,8477 & 22,50232  & 0,377727  \\ \hline
		GKD-b\_24\_n100\_m10 & 31,4582 & 22,81756  & 0,39744   \\ \hline
		GKD-b\_25\_n100\_m10 & 27,5302 & 10,32969  & 0,385703  \\ \hline
		GKD-b\_26\_n100\_m30 & 406,098 & 237,36841 & 1,89777   \\ \hline
		GKD-b\_27\_n100\_m30 & 334,907 & 207,80974 & 1,91974   \\ \hline
		GKD-b\_28\_n100\_m30 & 372,853 & 266,47381 & 1,95974   \\ \hline
		GKD-b\_29\_n100\_m30 & 345,314 & 207,86084 & 1,92148   \\ \hline
		GKD-b\_30\_n100\_m30 & 321,939 & 194,45926 & 1,93292   \\ \hline
		GKD-b\_31\_n125\_m12 & 35,7456 & 24,00046  & 0,517126  \\ \hline
		GKD-b\_32\_n125\_m12 & 30,7939 & 12,00497  & 0,496389  \\ \hline
		GKD-b\_33\_n125\_m12 & 29,9572 & 11,4256   & 0,515513  \\ \hline
		GKD-b\_34\_n125\_m12 & 37,8592 & 18,37087  & 0,526873  \\ \hline
		GKD-b\_35\_n125\_m12 & 40,6521 & 22,53968  & 0,518438  \\ \hline
		GKD-b\_36\_n125\_m37 & 405,815 & 250,38023 & 2,98219   \\ \hline
		GKD-b\_37\_n125\_m37 & 468,758 & 269,86338 & 2,6215    \\ \hline
		GKD-b\_38\_n125\_m37 & 517,218 & 329,25097 & 2,60684   \\ \hline
		GKD-b\_39\_n125\_m37 & 452,294 & 283,7038  & 2,54117   \\ \hline
		GKD-b\_40\_n125\_m37 & 549,944 & 371,75026 & 2,78845   \\ \hline
		GKD-b\_41\_n150\_m15 & 65,5224 & 42,17632  & 0,697358  \\ \hline
		GKD-b\_42\_n150\_m15 & 68,5533 & 41,7638   & 0,724217  \\ \hline
		GKD-b\_43\_n150\_m15 & 69,6583 & 42,90383  & 0,79368   \\ \hline
		GKD-b\_44\_n150\_m15 & 50,1342 & 24,19861  & 0,74847   \\ \hline
		GKD-b\_45\_n150\_m15 & 73,7269 & 45,95389  & 0,782823  \\ \hline
		GKD-b\_46\_n150\_m45 & 622,999 & 395,24969 & 3,50582   \\ \hline
		GKD-b\_47\_n150\_m45 & 552,181 & 323,5781  & 3,66742   \\ \hline
		GKD-b\_48\_n150\_m45 & 509,5   & 282,75466 & 3,50091   \\ \hline
		GKD-b\_49\_n150\_m45 & 643,86  & 417,45039 & 3,50226   \\ \hline
		GKD-b\_50\_n150\_m45 & 560,273 & 311,41638 & 3,53168   \\ \hline
    \end{tabular}
    \end{adjustbox}
    \caption{Resultados de la ejecución del algoritmo \textbf{AGG-posicion}}
\end{table}

\pagebreak
\pagebreak

\begin{table}[!ht]%
    \centering    
    \begin{adjustbox}{height=12cm}
    \begin{tabular}{|l|l|l|l|}
    \hline
        Caso & Coste~Medio~Obtenido & Desv & Tiempo~(s) \\ \hline
		GKD-b\_1\_n25\_m2    & 0       & 0         & 0,0691999 \\ \hline
		GKD-b\_2\_n25\_m2    & 0       & 0         & 0,0688575 \\ \hline
		GKD-b\_3\_n25\_m2    & 0       & 0         & 0,0690702 \\ \hline
		GKD-b\_4\_n25\_m2    & 0       & 0         & 0,0697938 \\ \hline
		GKD-b\_5\_n25\_m2    & 0       & 0         & 0,0692328 \\ \hline
		GKD-b\_6\_n25\_m7    & 27,352  & 14,63404  & 0,20604   \\ \hline
		GKD-b\_7\_n25\_m7    & 22,3541 & 8,25535   & 0,206522  \\ \hline
		GKD-b\_8\_n25\_m7    & 21,8265 & 5,06531   & 0,205248  \\ \hline
		GKD-b\_9\_n25\_m7    & 17,0692 & 0    & 0,221641  \\ \hline
		GKD-b\_10\_n25\_m7   & 31,4356 & 8,17037   & 0,227964  \\ \hline
		GKD-b\_11\_n50\_m5   & 11,4673 & 9,5412    & 0,148913  \\ \hline
		GKD-b\_12\_n50\_m5   & 8,23289 & 6,11185   & 0,147392  \\ \hline
		GKD-b\_13\_n50\_m5   & 9,5356  & 7,17329   & 0,145183  \\ \hline
		GKD-b\_14\_n50\_m5   & 11,9445 & 10,2813   & 0,148552  \\ \hline
		GKD-b\_15\_n50\_m5   & 8,80575 & 5,95262   & 0,149342  \\ \hline
		GKD-b\_16\_n50\_m15  & 42,7458 & 0     & 0,591303  \\ \hline
		GKD-b\_17\_n50\_m15  & 84,4087 & 36,30109  & 0,589358  \\ \hline
		GKD-b\_18\_n50\_m15  & 56,9453 & 13,74921  & 0,574655  \\ \hline
		GKD-b\_19\_n50\_m15  & 63,6934 & 17,28095  & 0,625478  \\ \hline
		GKD-b\_20\_n50\_m15  & 58,8834 & 11,16829  & 0,59067   \\ \hline
		GKD-b\_21\_n100\_m10 & 30,1887 & 16,35668  & 0,344962  \\ \hline
		GKD-b\_22\_n100\_m10 & 31,647  & 17,98266  & 0,330154  \\ \hline
		GKD-b\_23\_n100\_m10 & 16,4517 & 1,10632   & 0,336705  \\ \hline
		GKD-b\_24\_n100\_m10 & 32,1149 & 23,47426  & 0,335518  \\ \hline
		GKD-b\_25\_n100\_m10 & 22,7946 & 5,59409   & 0,348251  \\ \hline
		GKD-b\_26\_n100\_m30 & 203,94  & 35,21041  & 10,0877   \\ \hline
		GKD-b\_27\_n100\_m30 & 150,248 & 23,15074  & 10,4054   \\ \hline
		GKD-b\_28\_n100\_m30 & 164,853 & 58,47381  & 8,78707   \\ \hline
		GKD-b\_29\_n100\_m30 & 153,011 & 15,55784  & 10,1335   \\ \hline
		GKD-b\_30\_n100\_m30 & 151,439 & 23,95926  & 9,05302   \\ \hline
		GKD-b\_31\_n125\_m12 & 42,1172 & 30,37206  & 0,436014  \\ \hline
		GKD-b\_32\_n125\_m12 & 41,6994 & 22,91047  & 0,437212  \\ \hline
		GKD-b\_33\_n125\_m12 & 25,5398 & 7,0082    & 0,441545  \\ \hline
		GKD-b\_34\_n125\_m12 & 28,2153 & 8,72697   & 0,470765  \\ \hline
		GKD-b\_35\_n125\_m12 & 29,9754 & 11,86298  & 0,435588  \\ \hline
		GKD-b\_36\_n125\_m37 & 210,546 & 55,11123  & 23,9775   \\ \hline
		GKD-b\_37\_n125\_m37 & 273,008 & 74,11338  & 27,2155   \\ \hline
		GKD-b\_38\_n125\_m37 & 271,262 & 83,29497  & 29,0233   \\ \hline
		GKD-b\_39\_n125\_m37 & 226,107 & 57,5168   & 26,3883   \\ \hline
		GKD-b\_40\_n125\_m37 & 319,883 & 141,68926 & 27,5117   \\ \hline
		GKD-b\_41\_n150\_m15 & 31,9844 & 8,63832   & 0,696134  \\ \hline
		GKD-b\_42\_n150\_m15 & 46,8755 & 20,086    & 0,793278  \\ \hline
		GKD-b\_43\_n150\_m15 & 44,9803 & 18,22583  & 0,82751   \\ \hline
		GKD-b\_44\_n150\_m15 & 37,6753 & 11,73971  & 0,79      \\ \hline
		GKD-b\_45\_n150\_m15 & 38,9527 & 11,17969  & 0,868021  \\ \hline
		GKD-b\_46\_n150\_m45 & 332,74  & 104,99069 & 48,036    \\ \hline
		GKD-b\_47\_n150\_m45 & 359,598 & 130,9951  & 50,215    \\ \hline
		GKD-b\_48\_n150\_m45 & 331,59  & 104,84466 & 43,4513   \\ \hline
		GKD-b\_49\_n150\_m45 & 386,682 & 160,27239 & 51,1753   \\ \hline
		GKD-b\_50\_n150\_m45 & 304,535 & 55,67838  & 52,0638   \\ \hline
    \end{tabular}
    \end{adjustbox}
    \caption{Resultados de la ejecución del algoritmo \textbf{AGE-uniforme}}
\end{table}

\pagebreak
\pagebreak

\begin{table}[!ht]%
    \centering    
    \begin{adjustbox}{height=12cm}
    \begin{tabular}{|l|l|l|l|}
    \hline
        Caso & Coste~Medio~Obtenido & Desv & Tiempo~(s) \\ \hline
		GKD-b\_1\_n25\_m2    & 0,0000   & 0,00          & 7,32E-02 \\ \hline
		GKD-b\_2\_n25\_m2    & 0,0000   & 0,00          & 7,30E-02 \\ \hline
		GKD-b\_3\_n25\_m2    & 0,0000   & 0,00          & 7,30E-02 \\ \hline
		GKD-b\_4\_n25\_m2    & 0,0000   & 0,00          & 7,40E-02 \\ \hline
		GKD-b\_5\_n25\_m2    & 0,0000   & 0,00          & 7,37E-02 \\ \hline
		GKD-b\_6\_n25\_m7    & 38,9437  & 26,23         & 2,05E-01 \\ \hline
		GKD-b\_7\_n25\_m7    & 22,2649  & 8,17          & 2,05E-01 \\ \hline
		GKD-b\_8\_n25\_m7    & 29,1194  & 12,36         & 2,06E-01 \\ \hline
		GKD-b\_9\_n25\_m7    & 17,0692  & \textbf{0,00} & 2,23E-01 \\ \hline
		GKD-b\_10\_n25\_m7   & 28,1287  & 4,86          & 2,24E-01 \\ \hline
		GKD-b\_11\_n50\_m5   & 11,5860  & 9,66          & 1,51E-01 \\ \hline
		GKD-b\_12\_n50\_m5   & 8,5258   & 6,40          & 1,48E-01 \\ \hline
		GKD-b\_13\_n50\_m5   & 17,3503  & 14,99         & 1,48E-01 \\ \hline
		GKD-b\_14\_n50\_m5   & 8,1312   & 6,47          & 1,50E-01 \\ \hline
		GKD-b\_15\_n50\_m5   & 11,2965  & 8,44          & 1,48E-01 \\ \hline
		GKD-b\_16\_n50\_m15  & 78,5654  & 35,82         & 5,39E-01 \\ \hline
		GKD-b\_17\_n50\_m15  & 88,6213  & 40,51         & 5,32E-01 \\ \hline
		GKD-b\_18\_n50\_m15  & 43,1961  & 0,00          & 5,35E-01 \\ \hline
		GKD-b\_19\_n50\_m15  & 71,6372  & 25,22         & 5,34E-01 \\ \hline
		GKD-b\_20\_n50\_m15  & 74,4254  & 26,71         & 5,37E-01 \\ \hline
		GKD-b\_21\_n100\_m10 & 26,7693  & 12,94         & 3,18E-01 \\ \hline
		GKD-b\_22\_n100\_m10 & 37,0349  & 23,37         & 3,16E-01 \\ \hline
		GKD-b\_23\_n100\_m10 & 23,9095  & 8,56          & 3,18E-01 \\ \hline
		GKD-b\_24\_n100\_m10 & 36,4542  & 27,81         & 3,20E-01 \\ \hline
		GKD-b\_25\_n100\_m10 & 22,2168  & 5,02          & 3,18E-01 \\ \hline
		GKD-b\_26\_n100\_m30 & 179,2210 & 10,49         & 1,57E+00 \\ \hline
		GKD-b\_27\_n100\_m30 & 182,9370 & 55,84         & 1,56E+00 \\ \hline
		GKD-b\_28\_n100\_m30 & 231,5640 & 125,18        & 1,52E+00 \\ \hline
		GKD-b\_29\_n100\_m30 & 211,2390 & 73,79         & 1,51E+00 \\ \hline
		GKD-b\_30\_n100\_m30 & 221,3680 & 93,89         & 1,65E+00 \\ \hline
		GKD-b\_31\_n125\_m12 & 43,2570  & 31,51         & 3,90E-01 \\ \hline
		GKD-b\_32\_n125\_m12 & 38,1703  & 19,38         & 3,86E-01 \\ \hline
		GKD-b\_33\_n125\_m12 & 42,5201  & 23,99         & 3,85E-01 \\ \hline
		GKD-b\_34\_n125\_m12 & 29,6037  & 10,12         & 3,89E-01 \\ \hline
		GKD-b\_35\_n125\_m12 & 25,3283  & 7,22          & 3,90E-01 \\ \hline
		GKD-b\_36\_n125\_m37 & 198,7810 & 43,35         & 2,40E+00 \\ \hline
		GKD-b\_37\_n125\_m37 & 307,5010 & 108,61        & 2,17E+00 \\ \hline
		GKD-b\_38\_n125\_m37 & 310,1460 & 122,18        & 2,36E+00 \\ \hline
		GKD-b\_39\_n125\_m37 & 220,9880 & 52,40         & 2,30E+00 \\ \hline
		GKD-b\_40\_n125\_m37 & 266,6240 & 88,43         & 2,21E+00 \\ \hline
		GKD-b\_41\_n150\_m15 & 57,0299  & 33,68         & 5,31E-01 \\ \hline
		GKD-b\_42\_n150\_m15 & 54,9015  & 28,11         & 5,55E-01 \\ \hline
		GKD-b\_43\_n150\_m15 & 51,1365  & 24,38         & 5,36E-01 \\ \hline
		GKD-b\_44\_n150\_m15 & 64,7648  & 38,83         & 5,32E-01 \\ \hline
		GKD-b\_45\_n150\_m15 & 43,7700  & 16,00         & 5,33E-01 \\ \hline
		GKD-b\_46\_n150\_m45 & 490,2680 & 262,52        & 3,54E+00 \\ \hline
		GKD-b\_47\_n150\_m45 & 367,4580 & 138,86        & 3,48E+00 \\ \hline
		GKD-b\_48\_n150\_m45 & 324,5230 & 97,78         & 3,47E+00 \\ \hline
		GKD-b\_49\_n150\_m45 & 409,0530 & 182,64        & 3,42E+00 \\ \hline
		GKD-b\_50\_n150\_m45 & 363,4100 & 114,55        & 3,46E+00 \\ \hline
    \end{tabular}
    \end{adjustbox}
    \caption{Resultados de la ejecución del algoritmo \textbf{AGG-posicion}}
\end{table}

\pagebreak
\pagebreak

\begin{table}[!ht]%
    \centering    
    \begin{adjustbox}{height=12cm}
    \begin{tabular}{|l|l|l|l|}
    \hline
        Caso & Coste~Medio~Obtenido & Desv & Tiempo~(s) \\ \hline
		GKD-b\_1\_n25\_m2    & 0,0000   & 0,00           & 3,66E-01 \\ \hline
		GKD-b\_2\_n25\_m2    & 0,0000   & 0,00           & 3,67E-01 \\ \hline
		GKD-b\_3\_n25\_m2    & 0,0000   & 0,00           & 3,81E-01 \\ \hline
		GKD-b\_4\_n25\_m2    & 0,0000   & 0,00           & 3,83E-01 \\ \hline
		GKD-b\_5\_n25\_m2    & 0,0000   & 0,00           & 3,85E-01 \\ \hline
		GKD-b\_6\_n25\_m7    & 12,7180  & 0,00           & 2,46E+00 \\ \hline
		GKD-b\_7\_n25\_m7    & 14,0988  & 0,00           & 2,42E+00 \\ \hline
		GKD-b\_8\_n25\_m7    & 16,7612  & 0,00           & 2,37E+00 \\ \hline
		GKD-b\_9\_n25\_m7    & 17,0692  & \textbf{0,00}  & 2,43E+00 \\ \hline
		GKD-b\_10\_n25\_m7   & 23,2652  & \textbf{0,00}  & 2,39E+00 \\ \hline
		GKD-b\_11\_n50\_m5   & 1,9261   & \textbf{0,00}  & 1,92E+00 \\ \hline
		GKD-b\_12\_n50\_m5   & 2,0513   & \textbf{-0,07} & 2,10E+00 \\ \hline
		GKD-b\_13\_n50\_m5   & 2,3623   & 0,00           & 1,96E+00 \\ \hline
		GKD-b\_14\_n50\_m5   & 1,6632   & 0,00           & 1,99E+00 \\ \hline
		GKD-b\_15\_n50\_m5   & 2,8531   & 0,00           & 1,99E+00 \\ \hline
		GKD-b\_16\_n50\_m15  & 47,8949  & 5,15           & 1,19E+01 \\ \hline
		GKD-b\_17\_n50\_m15  & 48,1076  & \textbf{0,00}  & 1,16E+01 \\ \hline
		GKD-b\_18\_n50\_m15  & 43,1961  & 0,00           & 1,18E+01 \\ \hline
		GKD-b\_19\_n50\_m15  & 46,4125  & 0,00           & 1,25E+01 \\ \hline
		GKD-b\_20\_n50\_m15  & 47,7151  & \textbf{0,00}  & 1,18E+01 \\ \hline
		GKD-b\_21\_n100\_m10 & 14,8714  & 1,04           & 8,11E+00 \\ \hline
		GKD-b\_22\_n100\_m10 & 10,4533  & \textbf{-3,21} & 7,91E+00 \\ \hline
		GKD-b\_23\_n100\_m10 & 10,3482  & \textbf{-5,00} & 7,94E+00 \\ \hline
		GKD-b\_24\_n100\_m10 & 18,2991  & 9,66           & 7,73E+00 \\ \hline
		GKD-b\_25\_n100\_m10 & 18,1107  & 0,91           & 7,65E+00 \\ \hline
		GKD-b\_26\_n100\_m30 & 181,5200 & 12,79          & 5,44E+01 \\ \hline
		GKD-b\_27\_n100\_m30 & 209,3840 & 82,29          & 5,79E+01 \\ \hline
		GKD-b\_28\_n100\_m30 & 206,1600 & 99,78          & 5,48E+01 \\ \hline
		GKD-b\_29\_n100\_m30 & 179,8710 & 42,42          & 5,44E+01 \\ \hline
		GKD-b\_30\_n100\_m30 & 216,2250 & 88,75          & 5,46E+01 \\ \hline
		GKD-b\_31\_n125\_m12 & 21,2601  & 9,51           & 1,23E+01 \\ \hline
		GKD-b\_32\_n125\_m12 & 25,0131  & 6,22           & 1,21E+01 \\ \hline
		GKD-b\_33\_n125\_m12 & 21,2881  & 2,76           & 1,15E+01 \\ \hline
		GKD-b\_34\_n125\_m12 & 17,1852  & \textbf{-2,30} & 1,19E+01 \\ \hline
		GKD-b\_35\_n125\_m12 & 18,3822  & 0,27           & 1,13E+01 \\ \hline
		GKD-b\_36\_n125\_m37 & 166,0890 & 10,65          & 9,80E+01 \\ \hline
		GKD-b\_37\_n125\_m37 & 255,8350 & 56,94          & 9,41E+01 \\ \hline
		GKD-b\_38\_n125\_m37 & 280,7410 & 92,77          & 1,01E+02 \\ \hline
		GKD-b\_39\_n125\_m37 & 290,1770 & 121,59         & 1,06E+02 \\ \hline
		GKD-b\_40\_n125\_m37 & 245,6090 & 67,42          & 9,34E+01 \\ \hline
		GKD-b\_41\_n150\_m15 & 38,1517  & 14,81          & 1,85E+01 \\ \hline
		GKD-b\_42\_n150\_m15 & 43,6203  & 16,83          & 1,92E+01 \\ \hline
		GKD-b\_43\_n150\_m15 & 40,1703  & 13,42          & 1,85E+01 \\ \hline
		GKD-b\_44\_n150\_m15 & 33,4695  & 7,53           & 1,93E+01 \\ \hline
		GKD-b\_45\_n150\_m15 & 44,0248  & 16,25          & 1,83E+01 \\ \hline
		GKD-b\_46\_n150\_m45 & 276,2470 & 48,50          & 1,65E+02 \\ \hline
		GKD-b\_47\_n150\_m45 & 303,9200 & 75,32          & 1,68E+02 \\ \hline
		GKD-b\_48\_n150\_m45 & 333,9740 & 107,23         & 1,63E+02 \\ \hline
		GKD-b\_49\_n150\_m45 & 320,5940 & 94,18          & 1,50E+02 \\ \hline
		GKD-b\_50\_n150\_m45 & 290,3970 & 41,54          & 1,55E+02 \\ \hline
    \end{tabular}
    \end{adjustbox}
    \caption{Resultados de la ejecución del algoritmo \textbf{memetico-10-1.0}}
\end{table}

\pagebreak
\pagebreak

\begin{table}[!ht]%
    \centering    
    \begin{adjustbox}{height=12cm}
    \begin{tabular}{|l|l|l|l|}
    \hline
        Caso & Coste~Medio~Obtenido & Desv & Tiempo~(s) \\ \hline
		GKD-b\_1\_n25\_m2    & 0,0000   & 0,00           & 2,83E-01 \\ \hline
		GKD-b\_2\_n25\_m2    & 0,0000   & 0,00           & 2,83E-01 \\ \hline
		GKD-b\_3\_n25\_m2    & 0,0000   & 0,00           & 2,94E-01 \\ \hline
		GKD-b\_4\_n25\_m2    & 0,0000   & 0,00           & 2,92E-01 \\ \hline
		GKD-b\_5\_n25\_m2    & 0,0000   & 0,00           & 2,99E-01 \\ \hline
		GKD-b\_6\_n25\_m7    & 12,7180  & 0,00           & 1,68E+00 \\ \hline
		GKD-b\_7\_n25\_m7    & 14,0988  & 0,00           & 1,64E+00 \\ \hline
		GKD-b\_8\_n25\_m7    & 16,7612  & 0,00           & 1,58E+00 \\ \hline
		GKD-b\_9\_n25\_m7    & 17,0692  & \textbf{0,00}  & 1,56E+00 \\ \hline
		GKD-b\_10\_n25\_m7   & 23,2652  & \textbf{0,00}  & 1,67E+00 \\ \hline
		GKD-b\_11\_n50\_m5   & 1,9261   & \textbf{0,00}  & 1,15E+00 \\ \hline
		GKD-b\_12\_n50\_m5   & 2,0513   & \textbf{-0,07} & 1,27E+00 \\ \hline
		GKD-b\_13\_n50\_m5   & 3,4639   & 1,10           & 1,20E+00 \\ \hline
		GKD-b\_14\_n50\_m5   & 2,9222   & 1,26           & 1,16E+00 \\ \hline
		GKD-b\_15\_n50\_m5   & 2,8531   & 0,00           & 1,28E+00 \\ \hline
		GKD-b\_16\_n50\_m15  & 47,8949  & 5,15           & 8,02E+00 \\ \hline
		GKD-b\_17\_n50\_m15  & 48,1076  & \textbf{0,00}  & 8,03E+00 \\ \hline
		GKD-b\_18\_n50\_m15  & 52,0761  & 8,88           & 8,30E+00 \\ \hline
		GKD-b\_19\_n50\_m15  & 46,8501  & 0,44           & 7,76E+00 \\ \hline
		GKD-b\_20\_n50\_m15  & 47,7151  & \textbf{0,00}  & 7,62E+00 \\ \hline
		GKD-b\_21\_n100\_m10 & 19,3878  & 5,56           & 5,54E+00 \\ \hline
		GKD-b\_22\_n100\_m10 & 21,8986  & 8,23           & 5,57E+00 \\ \hline
		GKD-b\_23\_n100\_m10 & 19,3678  & 4,02           & 6,08E+00 \\ \hline
		GKD-b\_24\_n100\_m10 & 17,8188  & 9,18           & 5,49E+00 \\ \hline
		GKD-b\_25\_n100\_m10 & 20,6306  & 3,43           & 5,67E+00 \\ \hline
		GKD-b\_26\_n100\_m30 & 267,7840 & 99,05          & 4,45E+01 \\ \hline
		GKD-b\_27\_n100\_m30 & 240,2460 & 113,15         & 4,54E+01 \\ \hline
		GKD-b\_28\_n100\_m30 & 177,7250 & 71,35          & 4,38E+01 \\ \hline
		GKD-b\_29\_n100\_m30 & 232,7060 & 95,25          & 4,39E+01 \\ \hline
		GKD-b\_30\_n100\_m30 & 265,8580 & 138,38         & 4,42E+01 \\ \hline
		GKD-b\_31\_n125\_m12 & 17,3533  & 5,61           & 9,11E+00 \\ \hline
		GKD-b\_32\_n125\_m12 & 30,6868  & 11,90          & 8,64E+00 \\ \hline
		GKD-b\_33\_n125\_m12 & 29,0063  & 10,47          & 8,42E+00 \\ \hline
		GKD-b\_34\_n125\_m12 & 24,8794  & 5,39           & 8,56E+00 \\ \hline
		GKD-b\_35\_n125\_m12 & 23,9780  & 5,87           & 8,76E+00 \\ \hline
		GKD-b\_36\_n125\_m37 & 306,8180 & 151,38         & 7,64E+01 \\ \hline
		GKD-b\_37\_n125\_m37 & 340,7030 & 141,81         & 7,85E+01 \\ \hline
		GKD-b\_38\_n125\_m37 & 293,3280 & 105,36         & 7,81E+01 \\ \hline
		GKD-b\_39\_n125\_m37 & 304,1430 & 135,55         & 7,88E+01 \\ \hline
		GKD-b\_40\_n125\_m37 & 351,0650 & 172,87         & 8,24E+01 \\ \hline
		GKD-b\_41\_n150\_m15 & 39,4376  & 16,09          & 1,70E+01 \\ \hline
		GKD-b\_42\_n150\_m15 & 34,7406  & 7,95           & 1,62E+01 \\ \hline
		GKD-b\_43\_n150\_m15 & 45,7110  & 18,96          & 1,41E+01 \\ \hline
		GKD-b\_44\_n150\_m15 & 43,2451  & 17,31          & 1,50E+01 \\ \hline
		GKD-b\_45\_n150\_m15 & 37,8660  & 10,09          & 1,53E+01 \\ \hline
		GKD-b\_46\_n150\_m45 & 495,3060 & 267,56         & 1,42E+02 \\ \hline
		GKD-b\_47\_n150\_m45 & 358,2040 & 129,60         & 1,36E+02 \\ \hline
		GKD-b\_48\_n150\_m45 & 331,3880 & 104,64         & 1,38E+02 \\ \hline
		GKD-b\_49\_n150\_m45 & 390,7920 & 164,38         & 1,29E+02 \\ \hline
		GKD-b\_50\_n150\_m45 & 351,1920 & 102,34         & 1,33E+02 \\ \hline
    \end{tabular}
    \end{adjustbox}
    \caption{Resultados de la ejecución del algoritmo \textbf{memetico-10-0.1}}
\end{table}

\pagebreak
\pagebreak

\begin{table}[!ht]%
    \centering    
    \begin{adjustbox}{height=12cm}
    \begin{tabular}{|l|l|l|l|}
    \hline
        Caso & Coste~Medio~Obtenido & Desv & Tiempo~(s) \\ \hline
		GKD-b\_1\_n25\_m2    & 0,0000   & 0,00           & 2,87E-01 \\ \hline
		GKD-b\_2\_n25\_m2    & 0,0000   & 0,00           & 3,01E-01 \\ \hline
		GKD-b\_3\_n25\_m2    & 0,0000   & 0,00           & 2,89E-01 \\ \hline
		GKD-b\_4\_n25\_m2    & 0,0000   & 0,00           & 3,03E-01 \\ \hline
		GKD-b\_5\_n25\_m2    & 0,0000   & 0,00           & 3,07E-01 \\ \hline
		GKD-b\_6\_n25\_m7    & 13,4793  & 0,76           & 1,88E+00 \\ \hline
		GKD-b\_7\_n25\_m7    & 14,0988  & 0,00           & 1,65E+00 \\ \hline
		GKD-b\_8\_n25\_m7    & 16,7612  & 0,00           & 1,78E+00 \\ \hline
		GKD-b\_9\_n25\_m7    & 17,0692  & \textbf{0,00}  & 1,53E+00 \\ \hline
		GKD-b\_10\_n25\_m7   & 26,2380  & 2,97			 & 1,56E+00 \\ \hline
		GKD-b\_11\_n50\_m5   & 1,9261   & \textbf{0,00}  & 1,10E+00 \\ \hline
		GKD-b\_12\_n50\_m5   & 2,0513   & \textbf{-0,07} & 1,27E+00 \\ \hline
		GKD-b\_13\_n50\_m5   & 4,0205   & 1,66           & 1,37E+00 \\ \hline
		GKD-b\_14\_n50\_m5   & 5,7747   & 4,11           & 1,16E+00 \\ \hline
		GKD-b\_15\_n50\_m5   & 3,3944   & 0,54           & 1,13E+00 \\ \hline
		GKD-b\_16\_n50\_m15  & 42,7458  & 0,00           & 7,59E+00 \\ \hline
		GKD-b\_17\_n50\_m15  & 48,1076  & \textbf{0,00}  & 8,69E+00 \\ \hline
		GKD-b\_18\_n50\_m15  & 55,9383  & 12,74          & 8,01E+00 \\ \hline
		GKD-b\_19\_n50\_m15  & 46,4125  & 0,00           & 7,98E+00 \\ \hline
		GKD-b\_20\_n50\_m15  & 59,5468  & 11,83			 & 8,03E+00 \\ \hline
		GKD-b\_21\_n100\_m10 & 20,7321  & 6,90           & 5,55E+00 \\ \hline
		GKD-b\_22\_n100\_m10 & 23,9437  & 10,28          & 6,20E+00 \\ \hline
		GKD-b\_23\_n100\_m10 & 17,7771  & 2,43           & 5,76E+00 \\ \hline
		GKD-b\_24\_n100\_m10 & 17,9773  & 9,34           & 5,53E+00 \\ \hline
		GKD-b\_25\_n100\_m10 & 21,4260  & 4,23           & 5,47E+00 \\ \hline
		GKD-b\_26\_n100\_m30 & 206,0390 & 37,31          & 4,64E+01 \\ \hline
		GKD-b\_27\_n100\_m30 & 264,8930 & 137,80         & 4,46E+01 \\ \hline
		GKD-b\_28\_n100\_m30 & 186,7640 & 80,38          & 4,49E+01 \\ \hline
		GKD-b\_29\_n100\_m30 & 176,1700 & 38,72          & 4,52E+01 \\ \hline
		GKD-b\_30\_n100\_m30 & 213,5580 & 86,08          & 4,40E+01 \\ \hline
		GKD-b\_31\_n125\_m12 & 18,8318  & 7,09           & 9,33E+00 \\ \hline
		GKD-b\_32\_n125\_m12 & 26,3566  & 7,57           & 8,71E+00 \\ \hline
		GKD-b\_33\_n125\_m12 & 27,9774  & 9,45           & 9,02E+00 \\ \hline
		GKD-b\_34\_n125\_m12 & 30,1603  & 10,67          & 9,06E+00 \\ \hline
		GKD-b\_35\_n125\_m12 & 28,2190  & 10,11          & 8,75E+00 \\ \hline
		GKD-b\_36\_n125\_m37 & 202,5770 & 47,14          & 7,88E+01 \\ \hline
		GKD-b\_37\_n125\_m37 & 317,2610 & 118,37         & 7,89E+01 \\ \hline
		GKD-b\_38\_n125\_m37 & 317,1270 & 129,16         & 7,74E+01 \\ \hline
		GKD-b\_39\_n125\_m37 & 280,0200 & 111,43         & 8,06E+01 \\ \hline
		GKD-b\_40\_n125\_m37 & 267,6500 & 89,46          & 7,60E+01 \\ \hline
		GKD-b\_41\_n150\_m15 & 36,6471  & 13,30          & 1,45E+01 \\ \hline
		GKD-b\_42\_n150\_m15 & 39,8413  & 13,05          & 1,49E+01 \\ \hline
		GKD-b\_43\_n150\_m15 & 41,7757  & 15,02          & 1,49E+01 \\ \hline
		GKD-b\_44\_n150\_m15 & 39,4443  & 13,51          & 1,53E+01 \\ \hline
		GKD-b\_45\_n150\_m15 & 33,3907  & 5,62           & 1,50E+01 \\ \hline
		GKD-b\_46\_n150\_m45 & 408,3130 & 180,56         & 1,32E+02 \\ \hline
		GKD-b\_47\_n150\_m45 & 299,0410 & 70,44          & 1,37E+02 \\ \hline
		GKD-b\_48\_n150\_m45 & 270,6670 & 43,92          & 1,33E+02 \\ \hline
		GKD-b\_49\_n150\_m45 & 350,5850 & 124,18         & 1,32E+02 \\ \hline
		GKD-b\_50\_n150\_m45 & 428,0250 & 179,17         & 1,30E+02 \\ \hline
    \end{tabular}
    \end{adjustbox}
    \caption{Resultados de la ejecución del algoritmo \textbf{memetico-10-0.1best}}
\end{table}

\pagebreak
\pagebreak

\begin{table}[!ht]%
    \centering    
    \begin{adjustbox}{height=12cm}
    \begin{tabular}{|l|l|l|l|}
    \hline
        Caso & Coste~Medio~Obtenido & Desv & Tiempo~(s) \\ \hline
		GKD-b\_1\_n25\_m2    & 0,0000   & 0,00   & 2,92E-05 \\ \hline
		GKD-b\_2\_n25\_m2    & 0,0000   & 0,00   & 7,75E-06 \\ \hline
		GKD-b\_3\_n25\_m2    & 0,0000   & 0,00   & 7,34E-06 \\ \hline
		GKD-b\_4\_n25\_m2    & 0,0000   & 0,00   & 7,01E-06 \\ \hline
		GKD-b\_5\_n25\_m2    & 0,0000   & 0,00   & 7,01E-06 \\ \hline
		GKD-b\_6\_n25\_m7    & 31,0510  & 18,33  & 4,97E-02 \\ \hline
		GKD-b\_7\_n25\_m7    & 45,8746  & 31,78  & 5,26E-02 \\ \hline
		GKD-b\_8\_n25\_m7    & 36,2664  & 19,51  & 5,19E-02 \\ \hline
		GKD-b\_9\_n25\_m7    & 36,4972  & 19,43  & 4,98E-02 \\ \hline
		GKD-b\_10\_n25\_m7   & 30,2731  & 7,01   & 4,63E-02 \\ \hline
		GKD-b\_11\_n50\_m5   & 13,8704  & 11,94  & 1,16E-01 \\ \hline
		GKD-b\_12\_n50\_m5   & 12,3721  & 10,25  & 1,21E-01 \\ \hline
		GKD-b\_13\_n50\_m5   & 3,9208   & 1,56   & 1,12E-01 \\ \hline
		GKD-b\_14\_n50\_m5   & 16,8172  & 15,15  & 1,16E-01 \\ \hline
		GKD-b\_15\_n50\_m5   & 14,1281  & 11,27  & 1,18E-01 \\ \hline
		GKD-b\_16\_n50\_m15  & 140,1020 & 97,36  & 2,32E-01 \\ \hline
		GKD-b\_17\_n50\_m15  & 90,5339  & 42,43  & 2,13E-01 \\ \hline
		GKD-b\_18\_n50\_m15  & 128,7220 & 85,53  & 2,11E-01 \\ \hline
		GKD-b\_19\_n50\_m15  & 115,9870 & 69,57  & 2,13E-01 \\ \hline
		GKD-b\_20\_n50\_m15  & 62,5719  & 14,86  & 2,18E-01 \\ \hline
		GKD-b\_21\_n100\_m10 & 45,0814  & 31,25  & 3,57E-01 \\ \hline
		GKD-b\_22\_n100\_m10 & 43,5395  & 29,88  & 3,75E-01 \\ \hline
		GKD-b\_23\_n100\_m10 & 41,0927  & 25,75  & 3,61E-01 \\ \hline
		GKD-b\_24\_n100\_m10 & 34,4468  & 25,81  & 3,80E-01 \\ \hline
		GKD-b\_25\_n100\_m10 & 31,1789  & 13,98  & 3,70E-01 \\ \hline
		GKD-b\_26\_n100\_m30 & 288,8740 & 120,14 & 8,00E-01 \\ \hline
		GKD-b\_27\_n100\_m30 & 278,1840 & 151,09 & 7,99E-01 \\ \hline
		GKD-b\_28\_n100\_m30 & 354,2600 & 247,88 & 7,57E-01 \\ \hline
		GKD-b\_29\_n100\_m30 & 208,7300 & 71,28  & 8,23E-01 \\ \hline
		GKD-b\_30\_n100\_m30 & 290,0970 & 162,62 & 9,14E-01 \\ \hline
		GKD-b\_31\_n125\_m12 & 34,4464  & 22,70  & 4,84E-01 \\ \hline
		GKD-b\_32\_n125\_m12 & 39,9292  & 21,14  & 4,68E-01 \\ \hline
		GKD-b\_33\_n125\_m12 & 68,8769  & 50,35  & 4,66E-01 \\ \hline
		GKD-b\_34\_n125\_m12 & 54,4069  & 34,92  & 4,84E-01 \\ \hline
		GKD-b\_35\_n125\_m12 & 95,6531  & 77,54  & 4,66E-01 \\ \hline
		GKD-b\_36\_n125\_m37 & 324,7290 & 169,29 & 1,03E+00 \\ \hline
		GKD-b\_37\_n125\_m37 & 535,9820 & 337,09 & 1,11E+00 \\ \hline
		GKD-b\_38\_n125\_m37 & 249,6180 & 61,65  & 1,10E+00 \\ \hline
		GKD-b\_39\_n125\_m37 & 276,8610 & 108,27 & 1,06E+00 \\ \hline
		GKD-b\_40\_n125\_m37 & 352,4430 & 174,25 & 1,08E+00 \\ \hline
		GKD-b\_41\_n150\_m15 & 56,6003  & 33,25  & 5,98E-01 \\ \hline
		GKD-b\_42\_n150\_m15 & 62,7723  & 35,98  & 5,89E-01 \\ \hline
		GKD-b\_43\_n150\_m15 & 97,6482  & 70,89  & 6,00E-01 \\ \hline
		GKD-b\_44\_n150\_m15 & 32,2241  & 6,29   & 5,87E-01 \\ \hline
		GKD-b\_45\_n150\_m15 & 62,4453  & 34,67  & 5,83E-01 \\ \hline
		GKD-b\_46\_n150\_m45 & 631,4780 & 403,73 & 1,48E+00 \\ \hline
		GKD-b\_47\_n150\_m45 & 461,1270 & 232,52 & 1,46E+00 \\ \hline
		GKD-b\_48\_n150\_m45 & 561,8200 & 335,07 & 1,44E+00 \\ \hline
		GKD-b\_49\_n150\_m45 & 476,1300 & 249,72 & 1,44E+00 \\ \hline
		GKD-b\_50\_n150\_m45 & 553,8640 & 305,01 & 1,43E+00 \\ \hline
    \end{tabular}
    \end{adjustbox}
    \caption{Resultados de la ejecución del algoritmo \textbf{Enfriamiento~Simulado}}
\end{table}

\pagebreak
\pagebreak

\begin{table}[!ht]%
    \centering    
    \begin{adjustbox}{height=12cm}
    \begin{tabular}{|l|l|l|l|}
    \hline
        Caso & Coste~Medio~Obtenido & Desv & Tiempo~(s) \\ \hline
        GKD-b\_1\_n25\_m2    & 0,0000   & 0,00          & 2,17E-04 \\ \hline
        GKD-b\_2\_n25\_m2    & 0,0000   & 0,00          & 2,00E-04 \\ \hline
        GKD-b\_3\_n25\_m2    & 0,0000   & 0,00          & 1,82E-04 \\ \hline
        GKD-b\_4\_n25\_m2    & 0,0000   & 0,00          & 1,88E-04 \\ \hline
        GKD-b\_5\_n25\_m2    & 0,0000   & 0,00          & 1,79E-04 \\ \hline
        GKD-b\_6\_n25\_m7    & 15,2853  & 2,57          & 2,09E-03 \\ \hline
        GKD-b\_7\_n25\_m7    & 16,7547  & 2,66          & 1,94E-03 \\ \hline
        GKD-b\_8\_n25\_m7    & 21,2857  & 4,52          & 2,33E-03 \\ \hline
        GKD-b\_9\_n25\_m7    & 17,0692  & \textbf{0,00} & 2,00E-03 \\ \hline
        GKD-b\_10\_n25\_m7   & 23,2652  & \textbf{0,00} & 2,07E-03 \\ \hline
        GKD-b\_11\_n50\_m5   & 11,0456  & 9,12          & 2,13E-03 \\ \hline
        GKD-b\_12\_n50\_m5   & 5,3900   & 3,27          & 2,09E-03 \\ \hline
        GKD-b\_13\_n50\_m5   & 9,7842   & 7,42          & 1,91E-03 \\ \hline
        GKD-b\_14\_n50\_m5   & 8,6273   & 6,96          & 2,05E-03 \\ \hline
        GKD-b\_15\_n50\_m5   & 3,3944   & 0,54          & 2,34E-03 \\ \hline
        GKD-b\_16\_n50\_m15  & 85,8746  & 43,13         & 2,13E-02 \\ \hline
        GKD-b\_17\_n50\_m15  & 56,6031  & 8,50          & 2,06E-02 \\ \hline
        GKD-b\_18\_n50\_m15  & 78,9695  & 35,77         & 1,49E-02 \\ \hline
        GKD-b\_19\_n50\_m15  & 88,6846  & 42,27         & 1,81E-02 \\ \hline
        GKD-b\_20\_n50\_m15  & 60,2640  & 12,55         & 1,87E-02 \\ \hline
        GKD-b\_21\_n100\_m10 & 24,4649  & 10,63         & 1,87E-02 \\ \hline
        GKD-b\_22\_n100\_m10 & 31,2505  & 17,59         & 1,50E-02 \\ \hline
        GKD-b\_23\_n100\_m10 & 24,4460  & 9,10          & 1,69E-02 \\ \hline
        GKD-b\_24\_n100\_m10 & 26,3408  & 17,70         & 1,53E-02 \\ \hline
        GKD-b\_25\_n100\_m10 & 28,2012  & 11,00         & 1,51E-02 \\ \hline
        GKD-b\_26\_n100\_m30 & 261,9700 & 93,24         & 1,36E-01 \\ \hline
        GKD-b\_27\_n100\_m30 & 257,8800 & 130,78        & 1,49E-01 \\ \hline
        GKD-b\_28\_n100\_m30 & 259,3440 & 152,96        & 1,40E-01 \\ \hline
        GKD-b\_29\_n100\_m30 & 169,3520 & 31,90         & 1,47E-01 \\ \hline
        GKD-b\_30\_n100\_m30 & 210,2390 & 82,76         & 1,49E-01 \\ \hline
        GKD-b\_31\_n125\_m12 & 26,5903  & 14,85         & 3,22E-02 \\ \hline
        GKD-b\_32\_n125\_m12 & 37,6614  & 18,87         & 2,48E-02 \\ \hline
        GKD-b\_33\_n125\_m12 & 28,3203  & 9,79          & 2,58E-02 \\ \hline
        GKD-b\_34\_n125\_m12 & 37,5566  & 18,07         & 3,07E-02 \\ \hline
        GKD-b\_35\_n125\_m12 & 41,5648  & 23,45         & 3,12E-02 \\ \hline
        GKD-b\_36\_n125\_m37 & 295,3220 & 139,89        & 2,08E-01 \\ \hline
        GKD-b\_37\_n125\_m37 & 391,8620 & 192,97        & 2,08E-01 \\ \hline
        GKD-b\_38\_n125\_m37 & 370,9100 & 182,94        & 2,00E-01 \\ \hline
        GKD-b\_39\_n125\_m37 & 292,2150 & 123,62        & 2,15E-01 \\ \hline
        GKD-b\_40\_n125\_m37 & 348,6190 & 170,43        & 2,08E-01 \\ \hline
        GKD-b\_41\_n150\_m15 & 46,1376  & 22,79         & 5,53E-02 \\ \hline
        GKD-b\_42\_n150\_m15 & 62,4183  & 35,63         & 4,72E-02 \\ \hline
        GKD-b\_43\_n150\_m15 & 51,7077  & 24,95         & 6,09E-02 \\ \hline
        GKD-b\_44\_n150\_m15 & 46,4992  & 20,56         & 5,14E-02 \\ \hline
        GKD-b\_45\_n150\_m15 & 49,0046  & 21,23         & 4,69E-02 \\ \hline
        GKD-b\_46\_n150\_m45 & 481,2830 & 253,53        & 2,69E-01 \\ \hline
        GKD-b\_47\_n150\_m45 & 404,0910 & 175,49        & 2,68E-01 \\ \hline
        GKD-b\_48\_n150\_m45 & 447,9180 & 221,17        & 2,66E-01 \\ \hline
        GKD-b\_49\_n150\_m45 & 427,7620 & 201,35        & 2,52E-01 \\ \hline
        GKD-b\_50\_n150\_m45 & 466,2640 & 217,41        & 2,81E-01 \\ \hline
    \end{tabular}
    \end{adjustbox}
    \caption{Resultados de la ejecución del algoritmo \textbf{BMB}}
\end{table}

\pagebreak
\pagebreak

\begin{table}[!ht]%
    \centering    
    \begin{adjustbox}{height=12cm}
    \begin{tabular}{|l|l|l|l|}
    \hline
        Caso & Coste~Medio~Obtenido & Desv & Tiempo~(s) \\ \hline
        GKD-b\_1\_n25\_m2    & 0,0000   & 0,00          & 1,96E-04 \\ \hline
        GKD-b\_2\_n25\_m2    & 0,0000   & 0,00          & 1,74E-04 \\ \hline
        GKD-b\_3\_n25\_m2    & 0,0000   & 0,00          & 1,74E-04 \\ \hline
        GKD-b\_4\_n25\_m2    & 0,0000   & 0,00          & 1,80E-04 \\ \hline
        GKD-b\_5\_n25\_m2    & 0,0000   & 0,00          & 2,61E-04 \\ \hline
        GKD-b\_6\_n25\_m7    & 15,2853  & 2,57          & 2,31E-03 \\ \hline
        GKD-b\_7\_n25\_m7    & 24,0961  & 10,00         & 1,97E-03 \\ \hline
        GKD-b\_8\_n25\_m7    & 24,9729  & 8,21          & 2,19E-03 \\ \hline
        GKD-b\_9\_n25\_m7    & 32,9321  & 15,86         & 2,14E-03 \\ \hline
        GKD-b\_10\_n25\_m7   & 23,2652  & \textbf{0,00} & 1,83E-03 \\ \hline
        GKD-b\_11\_n50\_m5   & 7,2349   & 5,31          & 2,40E-03 \\ \hline
        GKD-b\_12\_n50\_m5   & 5,3900   & 3,27          & 2,29E-03 \\ \hline
        GKD-b\_13\_n50\_m5   & 10,5918  & 8,23          & 1,79E-03 \\ \hline
        GKD-b\_14\_n50\_m5   & 8,1201   & 6,46          & 2,19E-03 \\ \hline
        GKD-b\_15\_n50\_m5   & 10,7285  & 7,88          & 1,99E-03 \\ \hline
        GKD-b\_16\_n50\_m15  & 71,4951  & 28,75         & 1,87E-02 \\ \hline
        GKD-b\_17\_n50\_m15  & 96,6586  & 48,55         & 2,19E-02 \\ \hline
        GKD-b\_18\_n50\_m15  & 67,6253  & 24,43         & 1,69E-02 \\ \hline
        GKD-b\_19\_n50\_m15  & 66,8340  & 20,42         & 1,88E-02 \\ \hline
        GKD-b\_20\_n50\_m15  & 60,4311  & 12,72         & 1,77E-02 \\ \hline
        GKD-b\_21\_n100\_m10 & 15,6841  & 1,85          & 1,46E-02 \\ \hline
        GKD-b\_22\_n100\_m10 & 27,9863  & 14,32         & 1,41E-02 \\ \hline
        GKD-b\_23\_n100\_m10 & 21,5236  & 6,18          & 1,43E-02 \\ \hline
        GKD-b\_24\_n100\_m10 & 31,4815  & 22,84         & 1,33E-02 \\ \hline
        GKD-b\_25\_n100\_m10 & 27,1879  & 9,99          & 1,70E-02 \\ \hline
        GKD-b\_26\_n100\_m30 & 277,6290 & 108,90        & 1,14E-01 \\ \hline
        GKD-b\_27\_n100\_m30 & 256,9060 & 129,81        & 1,36E-01 \\ \hline
        GKD-b\_28\_n100\_m30 & 192,1070 & 85,73         & 1,47E-01 \\ \hline
        GKD-b\_29\_n100\_m30 & 201,1580 & 63,70         & 1,59E-01 \\ \hline
        GKD-b\_30\_n100\_m30 & 212,1870 & 84,71         & 1,42E-01 \\ \hline
        GKD-b\_31\_n125\_m12 & 21,9610  & 10,22         & 3,29E-02 \\ \hline
        GKD-b\_32\_n125\_m12 & 28,7789  & 9,99          & 2,63E-02 \\ \hline
        GKD-b\_33\_n125\_m12 & 32,5317  & 14,00         & 2,72E-02 \\ \hline
        GKD-b\_34\_n125\_m12 & 34,8429  & 15,35         & 3,11E-02 \\ \hline
        GKD-b\_35\_n125\_m12 & 30,5205  & 12,41         & 2,98E-02 \\ \hline
        GKD-b\_36\_n125\_m37 & 241,3160 & 85,88         & 2,01E-01 \\ \hline
        GKD-b\_37\_n125\_m37 & 368,4930 & 169,60        & 2,09E-01 \\ \hline
        GKD-b\_38\_n125\_m37 & 334,5600 & 146,59        & 2,08E-01 \\ \hline
        GKD-b\_39\_n125\_m37 & 328,0950 & 159,50        & 2,09E-01 \\ \hline
        GKD-b\_40\_n125\_m37 & 294,0100 & 115,82        & 2,15E-01 \\ \hline
        GKD-b\_41\_n150\_m15 & 46,3159  & 22,97         & 6,57E-02 \\ \hline
        GKD-b\_42\_n150\_m15 & 55,3973  & 28,61         & 5,72E-02 \\ \hline
        GKD-b\_43\_n150\_m15 & 53,6927  & 26,94         & 7,00E-02 \\ \hline
        GKD-b\_44\_n150\_m15 & 53,0782  & 27,14         & 5,48E-02 \\ \hline
        GKD-b\_45\_n150\_m15 & 43,3577  & 15,58         & 5,38E-02 \\ \hline
        GKD-b\_46\_n150\_m45 & 499,8240 & 272,07        & 2,84E-01 \\ \hline
        GKD-b\_47\_n150\_m45 & 423,3200 & 194,72        & 2,75E-01 \\ \hline
        GKD-b\_48\_n150\_m45 & 432,7870 & 206,04        & 2,78E-01 \\ \hline
        GKD-b\_49\_n150\_m45 & 510,9670 & 284,56        & 2,69E-01 \\ \hline
        GKD-b\_50\_n150\_m45 & 413,9570 & 165,10        & 2,37E-01 \\ \hline
    \end{tabular}
    \end{adjustbox}
    \caption{Resultados de la ejecución del algoritmo \textbf{ILS}}
\end{table}

\pagebreak
\pagebreak

\begin{table}[!ht]%
    \centering    
    \begin{adjustbox}{height=12cm}
    \begin{tabular}{|l|l|l|l|}
    \hline
        Caso & Coste~Medio~Obtenido & Desv & Tiempo~(s) \\ \hline
        GKD-b\_1\_n25\_m2    & 0,0000   & 0,00          & 9,03E-05 \\ \hline
        GKD-b\_2\_n25\_m2    & 0,0000   & 0,00          & 6,57E-05 \\ \hline
        GKD-b\_3\_n25\_m2    & 0,0000   & 0,00          & 6,53E-05 \\ \hline
        GKD-b\_4\_n25\_m2    & 0,0000   & 0,00          & 6,53E-05 \\ \hline
        GKD-b\_5\_n25\_m2    & 0,0000   & 0,00          & 6,50E-05 \\ \hline
        GKD-b\_6\_n25\_m7    & 21,6578  & 8,94          & 4,78E-02 \\ \hline
        GKD-b\_7\_n25\_m7    & 22,5656  & 8,47          & 4,19E-02 \\ \hline
        GKD-b\_8\_n25\_m7    & 28,1112  & 11,35         & 4,52E-02 \\ \hline
        GKD-b\_9\_n25\_m7    & 17,0692  & \textbf{0,00} & 4,29E-02 \\ \hline
        GKD-b\_10\_n25\_m7   & 26,8843  & 3,62          & 4,57E-02 \\ \hline
        GKD-b\_11\_n50\_m5   & 5,3757   & 3,45          & 1,17E-01 \\ \hline
        GKD-b\_12\_n50\_m5   & 10,5838  & 8,46          & 1,22E-01 \\ \hline
        GKD-b\_13\_n50\_m5   & 3,9208   & 1,56          & 1,12E-01 \\ \hline
        GKD-b\_14\_n50\_m5   & 8,0057   & 6,34          & 1,18E-01 \\ \hline
        GKD-b\_15\_n50\_m5   & 9,9887   & 7,14          & 1,16E-01 \\ \hline
        GKD-b\_16\_n50\_m15  & 61,8325  & 19,09         & 1,91E-01 \\ \hline
        GKD-b\_17\_n50\_m15  & 82,3942  & 34,29         & 1,82E-01 \\ \hline
        GKD-b\_18\_n50\_m15  & 81,4373  & 38,24         & 1,88E-01 \\ \hline
        GKD-b\_19\_n50\_m15  & 68,3885  & 21,98         & 1,89E-01 \\ \hline
        GKD-b\_20\_n50\_m15  & 62,6328  & 14,92         & 1,93E-01 \\ \hline
        GKD-b\_21\_n100\_m10 & 18,5641  & 4,73          & 3,02E-01 \\ \hline
        GKD-b\_22\_n100\_m10 & 20,7682  & 7,10          & 3,10E-01 \\ \hline
        GKD-b\_23\_n100\_m10 & 24,8474  & 9,50          & 3,19E-01 \\ \hline
        GKD-b\_24\_n100\_m10 & 25,0799  & 16,44         & 3,23E-01 \\ \hline
        GKD-b\_25\_n100\_m10 & 26,5692  & 9,37          & 3,24E-01 \\ \hline
        GKD-b\_26\_n100\_m30 & 265,2420 & 96,51         & 4,72E-01 \\ \hline
        GKD-b\_27\_n100\_m30 & 210,3720 & 83,27         & 4,85E-01 \\ \hline
        GKD-b\_28\_n100\_m30 & 220,6080 & 114,23        & 5,11E-01 \\ \hline
        GKD-b\_29\_n100\_m30 & 211,1980 & 73,74         & 5,31E-01 \\ \hline
        GKD-b\_30\_n100\_m30 & 220,9510 & 93,47         & 5,05E-01 \\ \hline
        GKD-b\_31\_n125\_m12 & 28,6885  & 16,94         & 3,72E-01 \\ \hline
        GKD-b\_32\_n125\_m12 & 27,5798  & 8,79          & 4,27E-01 \\ \hline
        GKD-b\_33\_n125\_m12 & 30,1468  & 11,62         & 3,71E-01 \\ \hline
        GKD-b\_34\_n125\_m12 & 29,5945  & 10,11         & 4,02E-01 \\ \hline
        GKD-b\_35\_n125\_m12 & 35,4123  & 17,30         & 4,11E-01 \\ \hline
        GKD-b\_36\_n125\_m37 & 262,2890 & 106,85        & 6,38E-01 \\ \hline
        GKD-b\_37\_n125\_m37 & 364,9430 & 166,05        & 5,61E-01 \\ \hline
        GKD-b\_38\_n125\_m37 & 434,3710 & 246,40        & 5,71E-01 \\ \hline
        GKD-b\_39\_n125\_m37 & 275,7830 & 107,19        & 5,90E-01 \\ \hline
        GKD-b\_40\_n125\_m37 & 309,0450 & 130,85        & 6,04E-01 \\ \hline
        GKD-b\_41\_n150\_m15 & 55,6954  & 32,35         & 4,80E-01 \\ \hline
        GKD-b\_42\_n150\_m15 & 48,4349  & 21,65         & 4,61E-01 \\ \hline
        GKD-b\_43\_n150\_m15 & 58,5323  & 31,78         & 4,57E-01 \\ \hline
        GKD-b\_44\_n150\_m15 & 41,2074  & 15,27         & 4,61E-01 \\ \hline
        GKD-b\_45\_n150\_m15 & 49,2451  & 21,47         & 4,96E-01 \\ \hline
        GKD-b\_46\_n150\_m45 & 422,0280 & 194,28        & 5,80E-01 \\ \hline
        GKD-b\_47\_n150\_m45 & 488,9690 & 260,37        & 5,62E-01 \\ \hline
        GKD-b\_48\_n150\_m45 & 454,4750 & 227,73        & 6,67E-01 \\ \hline
        GKD-b\_49\_n150\_m45 & 502,1520 & 275,74        & 6,09E-01 \\ \hline
        GKD-b\_50\_n150\_m45 & 587,1300 & 338,27        & 7,03E-01 \\ \hline
    \end{tabular}
    \end{adjustbox}
    \caption{Resultados de la ejecución del algoritmo \textbf{ILS-ES}}
\end{table}

\pagebreak

\bibliographystyle{plain}
\bibliography{references}

\vspace*{\fill}

\textit{%
El formato de este documento ha sido gracias a la plantilla de \LaTeX{} aqademia, cuyo autor es Atanasio Rubio Gil y la
cual se distribuye bajo la licencia GPL-2.0. En \url{https://github.com/moshidev/MH} se puede encontrar esta misma plantilla,
la cual tiene ligeras modificaciones hechas por mi con respecto a la original.
El repositorio con el código fuente original se encuentra en \url{https://gitlab.com/Groctel/aqademia}.
}

\end{document}