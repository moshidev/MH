Los métodos multiarranque buscan explorar el espacio de soluciones al
realizar distintas búsquedas locales a partir de distintas soluciones
generadas aleatoriamente, y así intentar disminuir las probabilidades
de quedar en un óptimo local, o al menos terminar seleccionando el mejor
óptimo local de entre las distintas búsquedas que se realicen.

\section{Esquema de la Búsqueda Multiarranque Básica}

Implementamos el siguiente esquema que depende en la búsqueda local
que describimos en el Capítulo 2. Consiste en realizar $n$ búsquedas
locales a partir de soluciones aleatorias, cada una de estas con un 
máximo de evaluaciones, y quedarse con la mejor solución que obtengamos.

\begin{lstlisting}[mathescape=true,caption={Esquema general de un algoritmo basada en Búsqueda Multiarranque Básica.},captionpos=b]
	var num_max_eval_por_busqueda

	def busqueda_multiarranque(numero_de_busquedas):
		var soluciones

		por cada búsqueda que queremos hacer:
			solucion_aleatoria = genera_solucion_aleatoria_y_factible()
			soluciones.inserta(busqueda_local(solucion_aleatoria, num_max_eval_por_busqueda))

		devuelve devuelve_mejor(soluciones)
\end{lstlisting}