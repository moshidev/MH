
El Problema de la Mínima Dispersión Diferencial es un problema de optimización combinatoria NP-Completo.\cite{Seminario2MH}

Este se presenta en distintas situaciones como al decidir la localización de instalaciones públicas,
la selección de grupos homogéneos, la identificación de grafos densos/regulares o el reparto equitativo en problemas de
flujo de red.\cite{DUARTE201546}

Intuitivamente estamos ante el Problema de la Mínima Dispersión Diferencial cuando dado un conjunto finito de puntos queremos
seleccionar un subconjunto de estos de forma que todos estén más o menos a la misma distancia unos de otros.

\section{Formulación del Problema}

Podemos describir el problema como
$$\textrm{Minimizar } Max_{i\in M}\{\sum_{j\in M}^{}d_{ij}\}-Min_{i\in M}\{\sum_{j\in M}^{}d_{ij}\} $$
Donde $M$ es un subconjunto de $N$, el conjunto de todos los posibles puntos y donde $d_{ij}$ es la distancia desde dos puntos $i$ y $j$.
$|M| = m$, siendo $m$ el número de elementos a escoger de $N$.

\section{Algoritmos para la resolución aproximada en tiempo polinómico}

La complejidad de este problema nos obliga a utilizar algoritmos que puedan encontrar el óptimo de una forma aproximada en un tiempo de computación razonable.
Entre las técnicas más populares podemos destacar GRASP, Búsqueda Local o métodos basados en Poblaciones.\cite{MDP2010}

Nosotros implementaremos entre otros algoritmos de tipo Greedy y de Búsqueda Local. Son algoritmos eficientes de por sí, pero para
mejorar la eficiencia de estos será necesario factorizar el cálculo de las soluciones a partir de otras soluciones de forma
que el algoritmo ejecute el número mínimo posible de operaciones al navegar por el espacio de soluciones.