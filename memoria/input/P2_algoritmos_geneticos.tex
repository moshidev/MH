En la segunda práctica se nos pide implementar un algoritmo genético con distintas variaciones,
según si el esquema de evolución es un esquema generacional con elitismo (AGG) o basada en
uno estacionario (AGE) y también según el operador de cruce, uniforme o basado en posición.
El esquema básico de un algoritmo se caracteriza por, sobre una población previamente inicializada
y en este orden, realizar operaciones sobre la población y con probabilidades preestablecidas
de seleccionar, cruzar mutar y reemplazar en cada iteración del algoritmo.

\section{Operadores comunes a los distintos algoritmos}

Como hemos introducido anteriormente, podemos definir una iteración de un algoritmo genético básico en pseudo-código
con los siguientes pasos:

\begin{minipage}{\textwidth}
\begin{lstlisting}[mathescape=true,caption={Esquema general de un algoritmo genético.},captionpos=b]
var cantidad_cromosomas_a_seleccionar
var num_genes
var probabilidad_cruce
var probabilidad_mutacion

poblacion = generar_poblacion_aleatoria(cantidad_cromosomas_a_seleccionar, num_genes)

def iteracion_algoritmo_genetico(poblacion):
	seleccionados = seleccionar(poblacion, cantidad_cromosomas_a_seleccionar)
	cruzar(seleccionados, probabilidad_cruce)
	mutar(seleccionados, probabilidad_mutacion)
	reemplazar(poblacion, seleccionados)
\end{lstlisting}
\end{minipage}

La definición de los algoritmos para cada uno de estos pasos puede ser diferente según el tipo de
algoritmo genético que diseñemos o implementemos. En este capítulo describiremos nuestra solución
y las diferentes definiciones de funciones para el algoritmo según las variaciones que se nos piden
para este.

\begin{minipage}{\textwidth}

	\subsection{Función de Selección}

	Utilizaremos para las distintas variantes selección por torneo binario. La selección por torneo binario
	consiste en escoger dos cromosomas de la población original al azar, compararlos entre sí y seleccionar el
	que mejor valoración tenga (en nuestro caso la solución con menor dispersión). Esto lo hacemos $n$ veces.
	
	En las variantes generacionales seleccionamos tantos como sea el tamaño de la población original, y en los
	estacionarios únicamente seleccionamos dos.
	
\begin{lstlisting}[mathescape=true,caption={Definición de la función de selección.},captionpos=b]
var generador_aleatorio_uniforme en rango {0, poblacion.size()-1}

def selecciona(poblacion, cantidad_a_seleccionar) -> poblacion_seleccionada:
	poblacion_seleccionada = []
	mientras que seleccionados < cantidad_a_seleccionar
		a = generador_aleatorio_uniforme.siguiente_numero()
		b = generador_aleatorio_uniforme.siguiente_numero()
		poblacion_seleccionada.anexa(a.dispersion < b.dispersion ? a : b)
	devuelve poblacion_seleccionada
\end{lstlisting}
\end{minipage}

\subsection{Función de cruce}

La función de cruce se encarga de seleccionar aleatoriamente parejas de soluciones a cruzar
con una probabilidad determinada. El cruce entre estas dos parejas se podrá realizar de distintas
formas, según veremos en secciones anteriores. El esquema general de la función de cruce es el siguiente.

\begin{minipage}{\textwidth}
\begin{lstlisting}[mathescape=true,caption={Esquema general de la función de cruce.},captionpos=b]
def cruce(poblacion, probabilidad, metodo_cruce):
	por cada pareja de la poblacion escogidas aleatoriamente:
		hijos = metodo_cruce(pareja.a, pareja.b)
		poblacion[pareja.a] = hijos.a
		poblacion[pareja.b] = hijos.b
\end{lstlisting}
\end{minipage}

Generar las parejas de forma aleatoria es un proceso muy costoso. En nuestro caso, ya que
anteriormente la selección se realizó aleatoriamente por torneo binario, no será necesario
volver a aleatorizar el vector que contiene la población de nuevo.

\begin{minipage}{\textwidth}
\begin{lstlisting}[mathescape=true,caption={Definición de la función de cruce.},captionpos=b]
def cruce(poblacion, probabilidad, metodo_cruce):
	cruces_por_hacer = probabilidad * poblacion.size()/2	# Esperanza matematica
	i = 0
	mientras i < cruces_por_hacer:
		# Cruzamos las parejas contiguas en el vector
		hijos = metodo_cruce(poblacion[i*2], poblacion[i*2+1])
		poblacion[i*2] = hijos.a
		poblacion[i*2+1] = hijos.b
		i = i + 1
\end{lstlisting}
\end{minipage}

\subsection{Función de mutación}

Es el tercer paso del esquema general de nuestro algoritmo genético y el que utilizaremos para las
distintas versiones que implementamos. Su función es modificar aleatoriamente algunos genes de
cada uno de los cromosomas para evitar que las soluciones se queden atascadas en zonas concretas
del espacio de búsqueda.

Nuestro operador de mutación selecciona cromosomas al azar, sobre los cuales selecciona uno de
sus genes para sustituirlo por otro al azar de entre los no pertenecientes a la solución (cromosoma).
Como hicimos en la función de cruce nos serviremos de la esperanza matemática para evitar generar números
aleatorios innecesarios.

La idea es la del siguiente listado.

\begin{minipage}{\textwidth}
\begin{lstlisting}[mathescape=true,caption={Definición de la función de mutación.},captionpos=b]
var numero_genes_por_solucion_valida
var dist_aleat_cromosoma en rango {0, poblacion.size()-1}
var dist_aleat_gen en rango {0, numero_genes_por_solucion_valida-1}

def muta(poblacion, probabilidad):
	# Esperanza matematica
	numero_mutaciones = probabilidad * numero_genes_por_solucion_valida
	
	por cada mutacion que tengamos que hacer:
		cromosoma_a_mutar = poblacion[dist_aleat_cromosoma.siguiente()]
		gen_a_mutar = dist_aleat_gen.siguiente()
		cromosoma_a_mutar.elimina(gen_a_mutar)
		cromosoma_a_mutar.inserta(gen aleatorio $\notin$ cromosoma_a_mutar)
\end{lstlisting}
\end{minipage}

\subsection{Función de reemplazamiento}

La función de reemplazamiento cambia completamente según la variación del algoritmo que hayamos
implementado. En secciones siguientes explicaremos la definición concreta del algoritmo que hayamos
utilizado.

El objetivo de la función de reemplazamiento es el de sustituir la población del inicio de la iteración
del algoritmo genético (o partes de esta) por la población modificada por los operadores vistos en subsecciones
anteriores (o partes de esta).

\section{Operadores de cruce}

Como dijimos anteriormente, la función de cruce se encarga de seleccionar aleatoriamente parejas de soluciones a cruzar
con una probabilidad determinada. Existen distintas técnicas para cruzar dos cromosomas.
Nosotros implementamos dos métodos distintos para cruzar dos soluciones, uno uniforme y otro basado en posición.

\subsection{Operador de cruce uniforme}

En nuestro problema, el MDD, el operador de cruce uniforme busca conservar las selecciones prometedoras manteniendo
la intersección de genes de ambos padres (tanto de genes en común como genes que no tienen ninguno de los dos). Para el
resto de selecciones que escapan la intersección los hijos se crean eligiendo aleatoriamente genes de un padre o del otro.

\begin{minipage}{\textwidth}
\begin{lstlisting}[mathescape=true,caption={Definición del operador de cruce uniforme.},captionpos=b]
def cruce_uniforme(padre_a, padre_b) -> hijo_a,hijo_b:
	hijo_a = padre_a $\cap $ padre_b
	hijo_b = padre_a $\cap $ padre_b
	por cada hijo:
		por cada gen no perteneciente a padre_a $\cap $ padre_b:	# padre_a $xor$ padre_b
			hijo.inserta_gen_al_50%_de_probabilidad(gen)
		repara(hijo)

	devuelve hijo_a,hijo_b
\end{lstlisting}
\end{minipage}

\begin{minipage}{\textwidth}

\subsubsection{Reparador para el operador uniforme}

Este operador genera soluciones que pueden no ser factibles, por tanto necesitaremos una función que repare a los hijos que genere
nuestro operador de cruce.

\begin{lstlisting}[mathescape=true,caption={Definición del operador de reparación para el operador de cruce uniforme.},captionpos=b]
def repara(cromosoma):
	si el cromosoma es factible:
		terminar
	mientras que el cromosoma tenga $mas$ genes de los que debería:
		gen = gen_perteneciente_al_cromosoma_que_maximiza_la_dispersion(cromosoma)
		cromosoma.elimina(gen)
	mientras que el cromosoma tenga $menos$ genes de los que debería:
		gen = gen_no_escogido_en_cromosoma_que_minimiza_la_dispersion(cromosoma)
		cromosoma.inserta(gen)
\end{lstlisting}
\end{minipage}

\subsection{Operador de cruce basado en posición}

El operador de cruce basado en posición al igual que el uniforme conserva en los hijos la intersección de los padres.
Se diferencia del uniforme en que en vez de ir escogiendo genes para los hijos aleatoriamente entre los padres distribuye
los genes de los padres que no intersectan entre los dos hijos (y por tanto no necesitando una función de reparación).
Es más disruptivo que el uniforme, y por tanto es más complicado que converja. \cite{Seminario3MH}
Definimos el operador de la siguiente forma.

\begin{minipage}{\textwidth}
\begin{lstlisting}[mathescape=true,caption={Definición del operador de cruce basado en posición.},captionpos=b]
def cruce_posicion(padre_a, padre_b) -> hijo_a,hijo_b:
	hijo_a = padre_a $\cap $ padre_b
	hijo_b = padre_a $\cap $ padre_b
	gen_xor = padre_a $\cup $ padre_b - padre_a $\cap $ padre_b
	aleatoriza(gen_xor)
	
	hijo_a.inserta(primera mitad gen_xor)
	hijo_b.inserta(segunda mitad gen_xor)

	devuelve hijo_a,hijo_b
\end{lstlisting}
\end{minipage}

\section{Modelos de los algoritmos genéticos implementados}

\subsection{Modelo generacional}

Consiste en seleccionar por torneo binario tantos cromosomas como cromosomas tenga la población original, cruzar
con una probabilidad $P_{c}$ (que en nuestro caso será de 0.7), mutar esta selección cruzada
una probabilidad $P_{m}$ (en nuestro caso 0.1) y reemplazar la población original por la nueva, aunque con elitismo,
esto es, manteniendo la mejor solución de la población original sobre el peor cromosoma de la nueva.
Definimos el operador de reemplazamiento del modelo generacional en el siguiente listado.

\begin{minipage}{\textwidth}
\begin{lstlisting}[mathescape=true,caption={Definición del operador de reemplazamiento del modelo generacional.},captionpos=b]
def reemplazamiento_generacional(a_reemplazar, poblacion_nueva):
	mejor_solucion = encuentra_mejor_solucion(a_reemplazar)
	swap(mejor_solucion, poblacion_nueva)
	a_reemplazar.elimina(encuentra_peor_solucion(poblacion_nueva))
	a_reemplazar.inserta(mejor_solucion)
\end{lstlisting}
\end{minipage}

\subsection{Modelo estacionario}

Por otra parte en el modelo estacionario en cada iteración del algoritmo genético seleccionamos únicamente dos
cromosomas, también por torneo binario. Cruzamos los cromosomas seleccionados con probabilidad 1 y mutamos con
probabilidad $P_{m}$ (en nuestro caso 0.1). A la hora de reemplazar la población original por la nueva únicamente
reemplazamos las peores soluciones de la original por las de la nueva únicamente si son mejores que esta. Podemos
verlo en la siguiente definición del algoritmo de reemplazamiento.

\begin{minipage}{\textwidth}
\begin{lstlisting}[mathescape=true,caption={Definición del operador de reemplazamiento del modelo estacionario.},captionpos=b]
def reemplazamiento_estacionario(a_reemplazar, poblacion_nueva):
	por cada cromosoma de la poblacion_nueva:
		peor_solucion = encuentra_peor_solucion(poblacion_nueva)
		si cromosoma.dispersion < peor_solucion.dispersion
			a_reemplazar.elimina(peor_solucion)
			a_reemplazar.inserta(cromosoma)
\end{lstlisting}
\end{minipage}