Los algoritmos de búsqueda local
pueden quedar atrapados en óptimos locales al nunca aceptar soluciones
peores de entre sus vecinos. Los algoritmos de Enfriamiento Simulado intentan
solventar este hecho de forma que aunque
siguen siendo algoritmos basadas en trayectorias solucionan en parte
este problema al aceptar vecinos potencialmente peores
en base a una función probabilística que depende en lo peor que sea la solución y
lo avanzado que nos encontremos en el proceso de la búsqueda.

\begin{minipage}{\textwidth}

	\section{Esquema del algoritmo implementado}

	El esquema general que seguimos para el algoritmo es el siguiente. Más adelante definiremos las
	funciones que invocamos pero que no definimos. En la implementación que adjuntamos a la memoria
	hacemos uso de la factorización de soluciones. Viene descrita en el Capítulo 2.

	\begin{lstlisting}[mathescape=true,caption={Esquema general de un algoritmo basado en el Enfriamiento Simulado.},captionpos=b]
var numero_maximo_de_evaluaciones
var numero_maximo_de_vecinos_explorados

def enfriamiento_simulado(solucion_inicial, temperatura_final):
	var solucion_actual = solucion_inicial
	var mejor_solucion = solucion_inicial
	var temperatura_inicial = calcula_temperatura_inicial(solucion_inicial)
	var temperatura_actual = temperatura_inicial
	var numero_de_enfriamientos_que_haremos =
		numero_maximo_de_evaluaciones/numero_maximo_de_vecinos_explorados

	mientras que temperatura_actual > temperatura_final y mientras que no superemos el numero maximo de evaluaciones en total:
		mientras que no superemos el número de vecinos generados y aceptados en cada iteración:	# L(T)
			var vecino_aleatorio = haz_vecino_aleatorio(solucion_actual)
			$\Delta f$ = vecino_aleatorio.dispersion() - solucion_actual.dispersion()
			# Si la solucion vecina es mejor o si la aceptamos segun la funcion de aceptacion_probabilistica
			si $\Delta f$ < 0 o aceptacion_probabilistica($\Delta f$, temperatura_actual):
				solucion_actual = vecino_aleatorio
				si solucion_actual.dispersion() < mejor_solucion.dispersion():
					mejor_solucion = solucion_actual
	
		temperatura_actual = politica_de_enfriamiento(temperatura_inicial, temperatura_actual, numero_de_enfriamientos_que_haremos)
	
	devuelve mejor_solucion
	\end{lstlisting}
\end{minipage}

Al seguir siendo un algoritmo basado en trayectorias podemos observar cierta similitud con una búsqueda local.
Escogemos un vecino (en nuestro caso aleatorio), comparamos este con la solución con la que estamos trabajando en este
momento y si es peor que la que acabamos de generar la sustituimos por la buena, ó si según la función de aceptación
probabilística queremos aceptar una solución un tanto peor para intentar descubrir un entorno mayor en el espacio de soluciones.
También destacamos el control que aplicamos sobre el número de vecinos que generamos y que aceptamos, denotándolo por L(T).
Lo explicaremos más adelante.

\begin{minipage}{\textwidth}
	\subsection{Cálculo de la temperatura inicial}
	
	Calculamos la temperatura inicial según la siguiente fórmula que se nos da en el guión de la práctica \cite{GuionPracticas3MH}.
	La implementamos en la definición de la función \texttt{calcula\_temperatura\_inicial(solucion\_inicial)}.

	$$ T_0 = \frac{\mu \cdot C(S_0)}{-ln(\varphi )} $$

	Tomando $\mu = \varphi = 0.3$ y $C(S_0)$ como el coste de nuestra solución inicial, tal y como se indica en el guión.
\end{minipage}

\begin{minipage}{\textwidth}
	\subsection{Velocidad del enfriamiento L(T)}

	En cada iteración del algoritmo guardamos constancia del número de vecinos que hemos generado, en este caso aleatoriamente,
	y del número de vecinos que hemos aceptado, ya sea por tener un coste menor o por la función de aceptación probabilística que
	explicamos y definimos más adelante.

	En nuestro caso, según cómo se indica en el guión de prácticas \cite{GuionPracticas3MH}, el número máximo de vecinos generados
	será igual a $10\cdot n$, siendo $n$ el tamaño del caso del problema, y el número máximo de soluciones aceptadas igual a
	$0.1\cdot m$, con $m$ el número máximo de vecinos generados.
\end{minipage}

\begin{minipage}{\textwidth}
	\subsection{Operador de mutación}
	
	Para generar un vecino aleatorio utilizamos el siguiente operador de mutación. Su propósito es el de eliminar un elemento aleatorio
	de la solución y sustituirlo por otro elemento igualmente aleatorio.

	\begin{lstlisting}[mathescape=true,caption={Operador de mutación para el algoritmo de Enfriamiento Simulado.},captionpos=b]
	def haz_vecino_aleatorio(solucion):
		var vecino_aleatorio = solucion

		vecino_aleatorio.elimina_gen_aleatoriamente()
		vecino_aleatorio.inserta_gen_aleatoriamente()

		devuelve vecino_aleatorio
	\end{lstlisting}
\end{minipage}

\begin{minipage}{\textwidth}
	\subsection{Función de aceptación probabilística}

	Según cómo definamos esta función tendremos mayor o menor probabilidad de aceptar una solución
	peor a la que estemos evaluando actualmente en función de su diferencia entre costes y de la temperatura
	actual. La función probabilística que propone Metrópolis(1953) para aceptar aumentos de temperatura
	en el sistema viene dada por $\exp (\frac{-\delta E}{k\cdot T})$, siendo $\delta E$ el incremento en energía,
	$k$ la constante de Boltzmann y $T$ la temperatura del sistema.\cite{Tema5MH} Aceptaremos o no una solución peor según la probabilidad
	que nos devuelve esta ecuación.

	\begin{lstlisting}[mathescape=true,caption={Función de aceptación probabilística para el algoritmo de Enfriamiento Simulado.},captionpos=b]
	$k = 1.380649E-23$	# Constante de Boltzmann
	var semilla
	var rgen = generador_de_numeros_aleatorios(semilla)
	
	def aceptacion_probabilistica($\Delta f$, temperatura_actual):
		var dist = distribucion uniforme de enteros en el rango $[0, 1]$
		var probabilidad_aceptacion = $1 / exp (\Delta f / (k\cdot $temperatura_actual$))$
		devuelve dist(rgen) < probabilidad_aceptacion
	\end{lstlisting}

	Comprobamos cómo si para obtener la probabilidad de aceptación utilizamos una ecuación
	más sencilla como $1 / exp (\Delta f / (temperatura\_actual))$ obtenemos peores resultados.
\end{minipage}

\begin{minipage}{\textwidth}
	\subsection{Política de enfriamiento}

	Existen distintas políticas de enfriamiento que podemos aplicar.
	Nosotros utilizaremos como política para la actualización de la temperatura actual en cada
	iteración del algoritmo el esquema de Cauchy modificado. Este nos permitirá controlar el número de iteraciones
	que realizará en total el algoritmo.

	\begin{lstlisting}[mathescape=true,caption={Función que define la política de enfriamiento para el algoritmo de Enfriamiento Simulado.},captionpos=b]
	def politica_de_enfriamiento(temperatura_inicial, temperatura_actual, numero_de_enfriamientos_que_haremos):
		var dividendo = temperatura_inicial - temperatura_final
		var divisor = numero_de_enfriamientos_que_haremos*temperatura_inicial*temperatura_actual
		var beta = dividendo/divisor
		devuelve temperatura_actual/(1 + beta*temperatura_actual)
	\end{lstlisting}
\end{minipage}