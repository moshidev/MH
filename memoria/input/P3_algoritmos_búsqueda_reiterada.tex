Los algoritmos basados en búsqueda reiterada consisten en realizar
una búsqueda por trayectorias como podría ser una búsqueda local
o una búsqueda por enfriamiento simulado, aplicar un operador de mutación
a la solución y seguir realizando búsquedas hasta que estimemos oportuno.

\begin{minipage}{\textwidth}
	\section{Esquema que utilizamos para implementar el ILS y el ILS-ES}

	El esquema para el algortimo para realizar la Búsqueda Local Reiterada
	y el híbrido entre ILS y Enfriamiento Simulado es el siguiente.
	Determinamos un número máximo de evaluaciones por iteración y un número
	máximo de iteraciones (búsquedas) que realizamos. Definimos más adelante
	el operador de mutación que utilizamos.

	\begin{lstlisting}[mathescape=true,caption={Esquema general de un algoritmo basado en Búsqueda Reiterada.
						Según apliquemos Búsqueda Local o Enfriamiento Simulado definimos la función de búsqueda
						tal y como describimos en sus respectivos capítulos.},captionpos=b]
		var num_max_eval_por_busqueda
		var probabilidad_de_mutacion

		def busqueda_reiterada(numero_de_busquedas):
			var mejor_solucion = genera_solucion_aleatoria_y_satisfacible()

			si numero_de_busquedas > 0:
				mejor_solucion = busqueda(mejor_solucion, num_max_eval_por_busqueda)
			por cada busqueda que nos quede por hacer:
				var solucion_mutada = muta(mejor_solucion, probabilidad_de_mutacion)
				var solucion_busqueda = busqueda(solucion_mutada, num_max_eval_por_busqueda)
				si solucion_busqueda es mejor que mejor_solucion:
					mejor_solucion = solucion_busqueda
			
			devuelve mejor_solucion
	\end{lstlisting}
\end{minipage}

\begin{minipage}{\textwidth}
\section{Operador de mutación}

Definimos como sigue el operador de mutación que utilizamos para el algoritmo de Búsqueda Reiterada
de esta práctica. Consiste en, según la probabilidad de mutación, sustituir proporcionalmente un
número de índices en la solución por otros completamente aleatorios.

\begin{lstlisting}[mathescape=true,caption={Definición del operador de mutación para nuestra implementación del algoritmo de Búsqueda Reiterada. Recordemos que la inserción y eliminación de elementos en una solucion está factorizada, y su definición la podemos encontrar en el Capítulo en el que describimos la Arquitectura de una Solución.},captionpos=b]
	def muta(a_mutar, probabilidad_de_mutacion):
		var numero_de_mutaciones = a_mutar.num_genes() * probabilidad_de_mutacion # Esperanza matemática
		var indices_escogidos = a_mutar.indices_genes()
		var indices_sin_escoger = genes_chart() / indices_escogidos

		por cada mutacion a realizar:
			var a_quitar = escoge_aleatorio(indices_escogidos)
			var a_insertar = escoge_aleatorio(indices_sin_escoger)
			a_mutar.elimina_punto_de_solucion(a_quitar)
			a_mutar.inserta_punto_en_solucion(a_insertar)
\end{lstlisting}
\end{minipage}